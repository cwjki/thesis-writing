\begin{opinion}
 En este trabajo se propone un sistema para la informatización de la gestión de varios procesos académicos de la facultad de Matemática y Computación de la Universidad de La Habana.  El uso de este sistema permitiría evitar errores y ahorrar tiempo.  Además, sienta las bases para poder incorporarles algoritmos de optimización que permitan hacer el proceso aún más cómodo.

 Para realizar esta tesis, Juan Carlos tuvo que estudiar contenidos que no forman parte de su plan de estudio, comprender el funcionamiento de los procesos que refleja en su tesis y actuar creativamente para resolver los problemas que aparecieron durante el proceso. Durante este tiempo de trabajo, Juan Carlos se ha caracterizado por su sobriedad, seriedad y responsabilidad.

 Por todo lo anterior, considero que estamos en presencia de un trabajo excelente, desarrollado por una persona que ha demostrado estar capacitada para desempeñarse como un excelente científico de la computación.

  \vspace{1cm}


  \begin{flushright}
    \underline{\hspace{6.5cm}}\\
    MSc. Fernando Raul Rodriguez Flores

    Facultad de Matemática y Computación
  
    Universidad de la Habana

    Noviembre, 2022
  \end{flushright}

  \begin{flushright}
   \underline{\hspace{6.5cm}}\\
   Lic. Alain Cartaya Salabarría

   Facultad de Matemática y Computación
  
   Universidad de la Habana

   Noviembre, 2022
 \end{flushright}
\end{opinion}