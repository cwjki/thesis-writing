\begin{resumen}
	En la facultad de Matemática y Computación de la Universidad de La Habana es necesario realizar un conjunto de
    procesos para administrar y planificar las actividades docentes. La mayoría de estos procesos se llevan a cabo
    de forma manual por los profesores, restándoles tiempo que pudieran destinar a la investigación
    u otras actividades más productivas.

    La asignación de docencia es una tarea que se realiza todos los semestres, y consiste en determinar los profesores  
    que deben impartir cada asignatura del semestre. 
    La planificación de las tesis se realiza cada año, y consiste en conformar los tribunales para todos los
    trabajos de culminación de estudios, y programar la defensa de los mismos.

    En este trabajo se implementa
    un sistema de gestión para informatizar estos dos procesos.
    Para ello, se diseña una base de datos relacional en la que se modelan los elementos que intervienen en 
    la asignación de docencia y la planificación de las tesis. 
    Se desarrolla una aplicación web que permite la posterior integración de la informatización y/o automatización 
    de otras tareas administrativas de la facultad.

    % Se crea una plataforma que facilite que posteriormente se informaticen y/o automaticen el resto de las tareas 
    % administrativas que se realizan en la facultad.



    \paragraph*{}
  \textbf{\emph{Palabras clave---}} sistema de gestión, aplicación web, diseño de base de datos.
\end{resumen}

\begin{abstract}
  Every year, a series of processes to manage and plan 
  docent activities are required in the Math and Computing 
  Science school at the University of Havana. 
  Most of these processes are being done manually and by 
  the teachers themselves, consuming time that 
  could be used for research and other more productive 
  activities.
  
  % Most of theses
  % processes are being implemented manually by the teachers,
  % consuming time that could be destined to research.

  Docent assigning is a task performed every semester, and
  consists in determining which teachers will teach which 
  subjects during the upcoming semester.
  The planning of the graduation thesis is performed every 
  year, and it consists in selecting the juries for all 
  the degree papers, and scheduling their presetations.
  
  In this work, a management system is implemented to computerize those two processes.
  For this purpose, a relational data base was designed, in which the elements that intervene in docent assigning and thesis planning are modeled.
  A web application was developed, to enable the integration 
  of future computerization or automation of other 
  management tasks.
  
  
  
  

% Every year, a set of processes to administrate and plan docent activities is required in the Math and Computing Science faculty at the University of Havana. Most of theses processes are being implemented manually by the teachers, consuming time that could be destined to research.

% Docent assigning is a task performed every semester, and consists in determining which teachers will teach which subject during the semester.
% The planning of the graduation thesis is performed every year, and it consists in selecting the juries for all the Graduation papers, and programming their defense.

% In this paper a management system is implemented to automate those 2 processes.
% For this purpose, a relational data base is designed, in which the elements that intervene in docent assigning and thesis planning are modeled.
% A web app was developed to enable the integration of the automation of other administrative tasks within the faculty.



	% In the Faculty of Mathematics and Computer Science of the University of Havana
  %   a set of processes are performed to manage and planning teaching activities.
  %   Most of these processes are carried out
  %   manually by the professors, causing them to waste time that they could dedicate to investigation.

  %   Teaching assignment is a task that has to be done every semester, and consists in determinate the professors
  %   who must teach each subject of the semester.
  %   Thesis planning process has to be done every year, and consists in design the tribunals for all
  %   the degree works, and schedule their defenses.

  %   In this work we developed a management system to computerize these two processes.
  %   For this, we designed a relational database for modelate the elements that intervene in
  %   teaching assignment and thesis planning. We developed a web application that allows the later integration 
  %   of computerization and/or automation of other administrative tasks of the faculty.

    \paragraph*{}
  \textbf{\emph{Keywords---}} management system, web application, database design.
\end{abstract}