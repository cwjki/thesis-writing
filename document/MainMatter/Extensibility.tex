\chapter{Extensibilidad}\label{chapter:extensibility}
Uno de los objetivos de este trabajo es la creación de una herramienta que 
permita la posterior integración de procesos de administración y planificación 
que se llevan a cabo en la Facultad de Matemática y Computación de la Universidad
de La Habana. 
En este capítulo propone una guía de cómo agregar nuevos 
procesos al sistema de gestión que se implementó.

\section{Estructura del cliente y servidor}
El desarrollo en el lado del servidor se llevó a cabo haciendo 
uso de la biblioteca Django, en particular Django Rest Framework. 
Por tanto la creación del proyecto que contiene el servidor se hizo 
con el [comando de django para crear el proyecto], como resultado 
se obtuvo la siguiente estructura para los ficheros del proyecto.

[DISTRIBUCION DE CARPETAS DEL SERVER]

Para la modelación de los procesos se crearon aplicaciones para separar 
el desarrollo de cada proceso, con el [comando de crear app de django rest framework]. 

Luego se crearon las apps teachingAssignment y thesisCommittee para la 

[AQUI EXPLICAR LA DISTRUBUCION DE LAS CARPETAS ECT]

\section{Recomendaciones para agregar un nuevo proceso en el sistema de gestión}
[SUPONER QUE VAMOS A AGREGAR UN NUEVO PROCESO, EXPLCIAR DONDE SE DEBEN CREAR LAS 
COSAS EN EL BACKEND Y EN EL FRONT]


Supongamos que se quiere agregar al sistema de gestión el proceso de planificación de 
las evaluaciones de un semestre. A continuación se describen los pasos principales
que se sugieren para la integración de este proceso.

\begin{itemize}
    \item Crear una nueva app de django con el comando tal.
    \item Dentro de la carpeta creada en el paso anterior definir los modelos, serializadores, 
    vistas y urls necesarios en ficheros nombrados models.py, serializers.py, views.py, urls.py respectivamente. 
    \item Crear una carpeta con el nombre del proceso que se está modelando dentro de las carpetas pages y components. 
    En la carpeta pages se encuentrar las vistar principales y dentro de la carpeta components definir las componentes 
    necesarias 
    \item En la carpeta services se implemetó una api para la comunicación con el servidor. Se creo un CRUDServiceFactory 
    que permite realizar pedidos de crear, listar, actualizar y borrar. Solo se necesita agregar la url y el tipo de dato 
    en el fichero resources. 

\end{itemize}



