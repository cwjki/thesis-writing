\chapter*{Introducción}\label{chapter:introduction}
\addcontentsline{toc}{chapter}{Introducción}

En la facultad de Matemática y Computación de la Universidad de La Habana (MATCOM)
se realizan un conjunto de procesos para administrar y
planificar las actividades docentes. La mayoría de 
estas tareas se llevan a cabo de forma manual, provocando que se 
cometan errores y se derroche tiempo que se pudiera emplear en 
investigaciones, ya que muchos de estos procesos se realizan por 
los profesores de la facultad. 

% que conlleva realizar 
% este tipo de tareas de forma manual
Con el objetivo de lidiar con estos problemas, en nuestra facultad se han desarrollado 
varias investigaciones que proponen algoritmos para automatizar algunos 
de estos procesos. Por ejemplo, la planificación de exámenes 
mundiales \cite{mundiales} y la organización de las defensas de tesis \cite{tribunales}.
% En cada una de estas investigaciones se propone una aplicación distinta 
% donde se resuelve un problema en particular.

Los procesos de asignación de docencia y planificación de 
las tesis continúan realizándose de forma manual. 
Todos los semestres, los jefes de cada departamento de la facultad deben realizar la asignación de docencia, 
que consiste en determinar qué profesores van a impartir las asignaturas correspondientes 
al semestre.
Por otra parte, como el ejercicio de culminación de estudios consiste en la realización 
de una tesis que debe ser defendida ante un tribunal de profesores, el encargado 
de planificar las tesis debe realizar anualmente, la confección de los tribunales y la programación de las defensas. 

Este trabajo se centra en informatizar la asignación de docencia y 
la planificación de las tesis. Para ello, se diseña una herramienta que permite 
la posterior integración de otros procesos administrativos de la facultad utilizando una única base de datos, ya que muchos 
comparten información.


El objetivo general de este trabajo es diseñar e implementar un sistema
que permita informatizar parte de la gestión de la facultad, en particular, 
la asignación de docencia y la planificación de las tesis.

Para cumplir con el objetivo general se plantean los siguientes objetivos 
específicos. 

\begin{itemize}
    \item Consultar literatura dedicada a informatizar y/o automatizar procesos de administración
    que se lleven a cabo en la facultad MATCOM. 
    \item Consultar literatura sobre las tecnologías para el desarrollo del software.
    \item Diseñar y modelar las entidades que intervienen en el proceso de asignación de docencia.
    \item Diseñar y modelar las entidades que intervienen en el proceso de planificación de las tesis.
    \item Implementar una aplicación web para informatizar la asignación de docencia y la 
    planificación de las tesis.
    
    \item Comprobar el funcionamiento del sistema simulando la asignación de docencia 
    y la planificación de las tesis correspondientes a un curso escolar. 
\end{itemize}

Se diseñó una base de datos en la que se almacena la información relacionada con los
departamentos, los profesores, las asignaturas y las tesis. A partir de esta información
se implementó una aplicación web con interfaces de usuario amigables, que reproducen el flujo 
de trabajo existente para realizar la asignación de docencia y la planificación de las tesis,
con el objetivo de facilitar la incorporación del sistema propuesto 
al flujo de trabajo de los profesores.

Este documento se compone de cinco capítulos. 
En el capítulo \ref{chapter:preliminaries} se describe el funcionamiento 
de la facultad, en particular, cómo se realizan los procesos de asignación 
de la docencia y planificación de las tesis. Además, se presentan las herramientas 
utilizadas para el desarrollo del software, especificando las características 
principales de las tecnologías utilizadas.

En el capítulo \ref{chapter:features} se describen las funcionalidades que 
se desean para que el sistema facilite la realización de la asignación
de docencia y la planificación de las tesis. 
El flujo de los procesos se ilustra con ejemplos, destacando en 
cuáles pasos se utiliza cada funcionalidad.

En el capítulo \ref{chapter:database} se presenta el diseño de la base 
de datos del sistema. Se describe la modelación de los procesos asignación de docencia y 
planificación de las tesis.

En el capítulo \ref{chapter:implementation} se presenta 
la aplicación web que se propone en este trabajo para gestionar la 
asignación de docencia y la planificación de las tesis. Se describen 
los pasos que se deben seguir para realizar cada uno de estos procesos 
a través del sistema. Además, se presenta una funcionalidad implementada 
para salvar el estado de la base de datos en documentos CSV y poblarla 
a partir de la información almacenada en los mismos. 

Como uno de los objetivos del trabajo es la posterior integración de procesos 
administrativos, en el capítulo \ref{chapter:extensibility}
se proponen los pasos y pautas a seguir para incorporar nuevos procesos al 
sistema. Además, se presenta la organización y estructura del 
proyecto y se describen algunos detalles de implementación. 




