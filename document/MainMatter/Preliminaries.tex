\chapter{Preliminares}\label{chapter:preliminaries}
Este capítulo provee una breve introducción a los distintos 
procesos que se llevan a cabo en un departamento de la Facultad
de Matemática y Computación de La Universidad de La Habana, así
como una descripción de las herramientas utilizadas en la 
solución brindada. 

\section{Funcionamiento de la facultad}
Cada año en la Facultad de Matemática y Computación de La 
Universidad de La Habana se realizan un conjunto de procesos de 
planificación y administración relacionados con los cursos escolares.
La ejecución de los procesos es delegada a los distintos
departamentos de la facultad y en su mayoría, si no en su totalidad,
se realizan actualmente de manera manual y no informatizada,
por ejemplo:

\begin{itemize}
    \item Asignación de docencia
    \item Confección de los tribunales de tesis 
    \item Asignación de cursos optativos
    \item Asignación de alumnos ayudantes
    \item Planificación de las evaluaciones del semestre
\end{itemize}


A continuación se describen los procesos de asignación de docencia y 
confección de tribunales de tesis que son llevados
a cabo en la Facultad de Matemática y Computación de La 
Universidad de La Habana:

\subsection{Asignación de docencia}
El proceso de asignación de docencia en un departamento consiste en, 
dadas las asignaturas a impartir en un semestre y el conjunto de 
profesores del mismo, realizar una distribución que satisfaga los
requerimientos de dichas asignaturas, contemplando la experiencia y 
las preferencias de los profesores.

Cada asignatura tiene sus particularidades, como la cantidad de horas
a impartir en conferencias, clases prácticas, seminarios, laboratorios u
otras modalidades, y la cantidad de grupos de clase que reciben dicha 
actividad. Estas particularidades varían en dependencia de la matrícula 
del curso y del plan de estudio al que la asignatura pertezca.

Por ejemplo para la asignatura Modelos de Optimización I, perteneciente
al plan de estudio D, para la carrera de Ciencia de la Computación que 
se imparte en el tercer año, se tiene un total de horas a impartir de 64,
de esas, 32 horas son de conferencia y 32 horas son de clases prácticas,
además se tiene que las conferencias se imparten a un solo grupo mientras
que las clases prácticas a dos.

(No se si agregar algo sobre esto)
Actualmente este proceso es ejecutado por una persona 'ciclano' y debe
ponerse de acuerdo con los profes, muchas veces.


\subsection{Tribunales de tesis}
Para realizar la confección de un tribunal de tesis
se necesitan definir tres incógnitas fundamentales: el local,
la fecha y el tribunal. El tribunal debe estar compuesto por tres
profesores que asuman los roles de secretario, presidente y 
oponente. Los profesores que integren el tribunal no pueden ser 
tutores o cotutores de la tesis en cuestión, y deben tener dominio
sobre el tema que será presentado.
Para la elección de la fecha y el local se debe tener en cuenta que la 
defensa de una tesis dura aproximadamente 45 minutos.

(No se si agregar algo sobre esto)
lo mismo que en el proceso anterior, mencionar las desventajas de 
hacer este proceso no informatizado




\section{Descripción del software}
Para la informatización de estos procesos se implementa 
un sistema de gestión a través de una aplicación web,
utilizando un modelo cliente-servidor.
Para el desarrollo en el lado del servidor se hace uso del framework Django, en particular Django Rest Framework, aprovechando la
versatilidad, seguridad, escalabilidad y mantenibilidad
que el framework ofrece.
Mientras que en el lado del cliente se hace uso
del framework Quasar, que se basa en Vue.js, creando una
single-page application (SPA) o aplicación de página única.
A continuación se profundiza en el modelo 
y herramientas utilizadas.

\subsection{Modelo cliente-servidor}
El modelo cliente-servidor es uno de los más populares en
el mundo de la informática actual, es utilizado para el 
desarrollo de todo tipo de aplicaciones, incluso muchos 
de los protocolos utilizados a diario implementan este modelo,
tales como File Transfer Protocol (FTP), Simple Mail Transfer Protocol
(SMTP) y Hypertext Transfer Protocol (HTTP).

El modelo cliente-servidor puede ser definido como una arquitectura
de software compuesta por los proveedores de un recurso o servicio, 
denominados servidores, y los solicitantes del servicio, denominados
clientes. En este modelo se realiza una comunicación de procesos
que implica el intercambio de datos tanto por parte del cliente 
como del servidor, realizando cada uno funciones diferentes. La comunicación
se realiza frecuentemente a través de una red informática, pero 
tanto el cliente como el servidor pueden residir en un mismo sistema.

Entre las principales ventajas que ofrece este modelo se encuentran:

\begin{itemize}
    \item centralización de los recursos, los accesos y la integridad de los datos son controlados por el servidor de forma que un programa cliente defectuoso o no autorizado no pueda dañar el sistema.
    \item división del procesamiento de la aplicación en varios dispositivos.
    \item permite la escalabilidad del sistema, en cualquier momento se puede mejorar la capacidad de procesamiento aumentando la cantidad de servidores sin que el funcionamiento de la red se vea afectado.
    \item la separación en cliente y servidor permite el uso de tecnologías enfocadas en las labores específicas que realiza cada uno. 
\end{itemize}
 


\subsection{Desarrollo del servidor}
Para la implementación de los servicios que consumen los clientes del 
sistema de gestión se hace uso del framework Django. Django es un framework
web de alto nivel, gratuito y de código abierto que permite el desarrollo rápido de 
sitios web seguros y mantenibles. Entre las principales características
del framework se encuentran:

\begin{itemize}
    \item Django sigue la filosofía
    de diseño DRY (Don't Repeat Yourself), brindando un conjunto de funcionalidades implementadas
    por los desarrolladores para evitar la repetición de código en procesos
    comunes en el desarrollo web.
    \item Django implementa el patrón de diseño Model-View-Template (MVT), que
    consta de tres componentes esenciales Modelo, Vista,
    Plantilla. Estas componentes son responsables de diferentes
    partes del código e incluso pueden utililizarse de forma independiente. Al ser MVT una 
    ``loosely coupled architecture'' o una ``arquitectura
    débilmente acoplada'', permite de manera sencilla la integración de componentes
    desarrolladas en paralelo.
    \item Django está implementado en Python, por lo que cuenta con un 
    extenso conjunto de bibliotecas para resolver distintas tareas. Entre
    las bibliotecas más utilizadas resalta Django REST framework, desarrollada 
    para la creación de interfaces de programación de aplicaciones (APIs).
    \item Django proporciona un Object Relational Mapper (ORM) o mapeador
    relacional de objetos, permitiendo la interacción con la base de datos
    de forma orientada a objetos. Brinda la posibilidad de crear tablas, insertar,
    editar, borrar y extraer datos sin escribir consultas SQL, acelerando el proceso
    de desarrollo web. 
    \item Django sigue también la filosofía de ``Batteries Included'' o ``Baterías Incluidas''
    proporcionando una biblioteca estándar rica y versátil con herramientas para
    crear sistemas complejos sin la necesidad de instalar paquetes separados. Algunas de 
    estas ``baterías'' son ``Django Admin'', ``Django ORM'', ``Authentication'' y
    ``HTTP''.
    \item Django cuenta con extensa documentación, desde la documentación oficial hasta 
    contenido en forma de artículos, tutoriales y cursos en línea.
    \item Django es escalable, utiliza una arquitectura de ``shared-nothing'' o ``nada compartido'',
    que permite agregar hardware en cualquier nivel: servidores de bases de datos,
    servidores de almacenamiento en caché o servidores web (REFERENCIA).

\end{itemize}

 


\subsection{Desarrollo del cliente}
Dentro de las tecnologías de desarrollo web más utilizadas, sin duda, 
JavaScript es una de las más populares en la actualidad. JavaScript es 
un lenguaje de programación ligero, interpretado o compilado ``just-in-time''
o ``justo-a-tiempo'' con funciones de primera clase. Es un lenguaje de programación
basado en prototipos, multiparadigma, de un solo hilo, dinámico, con
soporte para programación orientada a objetos, imperativa y declarativa.
Se usa principalmente del lado del cliente, implementado como parte de un navegador 
web permitiendo mejorar las interfaces de usarios y la creación de
páginas web dinámicas[REFERENCIA WIKI].

La popularidad de JavaScript para el desarrollo web ha impulsado
la creación de distintos frameworks, entre los más 
utilizados se encuentran:

\subparagraph{Angular}
Lanzado en 2010, Google está a cargo de desarrollar y 
mantener Angular. La principal filosofía tras Angular es 
tener todo lo que el desarrollador necesita, es un framework
con muchas características integradas, tales como: validaciones 
de formularios, envíos de solicitudes http, enrutamiento y 
manejo de estados, cuenta además con herramientas adicionales
como: línea de comandos, interfaces para el manejo y 
creación de proyectos. Debido a esto muchos han llamado a Angular
una plataforma, más que un framework. Google y Wix se 
encuentran entre las empresas más populares que utilizan
Angular.

\subparagraph{React}
Lanzado en 2013, Facebook desarrolla y mantiene React.
A diferencia de Angular, React se enfoca en ser lo más minimalista posible,
especializada en el desarrollo de interfaces de usuario, describiéndose
a sí misma como una biblioteca. Cuenta con muchas menos funciones integradas
que Angular por lo que se terminan utilizando muchos paquetes o
dependencias de terceros al desarrollar un proyecto, ya sea para enrutamiento, manejo de estados o
envío de solicitudes http.
Whatsapp, Instagram, Paypal, Glassdoor y BBC son algunas de 
las compañías populares que usan React.

\subparagraph{Vuejs}
Lanzado en 2014, su desarrollo y mantenimiento es llevado a 
cabo por un equipo de colaboradores a nivel mundial, siendo  
Evan Yu, su principal creador, un ex ingeniero de Google.
Se puede decir que Vue se encuentra entre Angular y React en cuanto 
a la cantidad de funciones integradas que ofrece, más que React, pero 
menos que Angular. Sitios web como GitLab y Alibaba están usando Vue. \\






Para el desarrollo del front-end del sistema de gestión 
para el funcionamiento de un departamento docente en la facultad
de Matemática y Computación de la UH se 


\subparagraph{Quasar}
Quasar es un marco basado en Vue de alto rendimiento que le 
permite crear rápidamente aplicaciones de una sola página
(SPA), aplicaciones renderizadas del lado del servidor (SSR)
, aplicaciones móviles y más. Este marco consiste en un gran
conjunto de herramientas con cientos de componentes Vue,
directivas, componibles y más.




