\chapter{Preliminares}\label{chapter:preliminaries}


\section{Funcionamiento de la facultad}
En todas las instituciones se realizan un conjunto de procesos .....
Algunos de estos procesos que son llevados a cabo en nuestra facultad son 
los siguientes:

\subsection{Asignación de docencia}
El proceso de asignación de docencia en un departamento consiste, en 
dadas las asignaturas a impartir en un semestre y el conjunto de 
profesores del mismo, realizar una distribucion que satisfaga los
requerimientos de dichas asignaturas, contemplando la disponibilidad
de los profesores (mencionar algo de la capacidad y experiencia de los 
profes).

Cada asignatura tiene algunas particularidades como la cantidad de horas
a impartir en conferencias, clases prácticas, seminarios, laboratorios u
otras modalidades, y la cantidad de grupos de clase que reciben dicha 
actividad. Estas particularidades varían en dependencia de la matrícula 
del curso y del plan de estudio al que la asignatura pertezca.

Por ejemplo para la asignatura 'Modelos de Optimización I', perteneciente
al plan de estudio D, para la carrera de Ciencias de la Computación que 
se imparte en el tercer año, se tiene un total de horas a impartir de 64,
de esas, 32 horas son de conferencia y 32 horas son de clases prácticas,
además se tiene que las conferencias se imparten a un solo grupo mientras
que las clases prácticas a dos.

\subsection{Tribunales de tesis}
Para realizar la confección de un tribunal de tesis es necesario la
asignación de tres profesores para que asuman los roles de secretario,
presidente y oponente. Además se necesita fijar una fecha y un lugar para 
la exposición de la misma.
Los profesores que integren el tribunal no pueden ser tutores o cotutores
de la tesis en cuestión y deben tener dominio sobre el tema que será 
presentado.
Para la elección de la fecha y el local se debe tener en cuenta que la 
defensa de una tesis dura aproximadamente 45 minutos.

\section{Breve descripción de las herramientas que se usan}
Para la informatización de estos procesos se implementó
una aplicación web, con una arquitectura cliente-servidor.
Para el desarrollo del back-end se hizo uso del framework Django, en particular Django Rest Framework, aprovechando la
versatilidad, seguridad, escalabilidad y mantenibilidad
que el framework ofrece.
En el front-end se implementó una aplicación web utilizando el framework Quasar, el cual está a su vez basado en Vue.js, creando una (SPA) fácil de mantener
gracias a las componentes y características reactivas
del framework.

