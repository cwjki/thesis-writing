\chapter{Preliminares}\label{chapter:preliminaries}
En este capítulo se presenta una introducción a los procesos que se llevan a cabo en la Facultad
de Matemática y Computación de La Universidad de La Habana, así
como una descripción de las herramientas utilizadas 
en este trabajo. 

En la sección \ref{section:funcionamiento de la facultad} se describe el funcionamiento 
de la facultad y se  y en la mas cual.

\section{Funcionamiento de la facultad}\label{section:funcionamiento de la facultad}


Cada año en la Facultad de Matemática y Computación de La 
Universidad de La Habana se realizan procesos administrativos y de 
planificación relacionados con los cursos escolares. A continuación se describen 
algunos de estos procesos. \\

[SI SE MANTIENE ESTO, REVISAR LA REDACCION DE LOS PROCESOS.] \\

\paragraph{Asignación de docencia:}
Consiste en dadas las asignaturas a impartir en el semestre actual,
realizar una distribución de los profesores del departamento, tal que se 
cumplan ciertas restricciones. Por ejemplo, es necesario impartir
cierta cantidad de horas de conferencia, clases prácticas u otro tipo de actividad 
relacionada con cada asignatura, de acuerdo al plan de estudio al que pertenezca.

\paragraph{Planificación de las evaluaciones en el semestre:}
Por cada asignatura que se imparte en el semestre es necesario realizar un planificación de 
las evaluaciones del curso. Por ejemplo, se deben definir la cantidad de evaluaciones, el tipo 
de evaluación: trabajos de control (TC), intrasemestral, preguntas escritas, proyectos, seminarios, entre 
otros y las fechas en las que se efectuarán estas evaluaciones.

\paragraph{Planificación de los exámenes finales y las revalorizaciones:}
Por cada asignatura es necesario realizar la planificación de los exámenes
finales y revalorizaciones. Este es un proceso complejo, se debe tener en cuenta 
varios factores como la cantidad de locales, con sus respectivas capacidades,
todas las asignaturas no tiene las mismas características, por ejemplo la asignatura de 
Programación requiere de dos días completos consecutivos en el laboratorio para 
la realización de sus exámenes. En el caso de las revalorizaciones
se debe tener en cuenta las asignaturas que tengan estudiantes comunes para no programar los 
exámenes en los mismos horarios. Además se pueden valorar otros factores como el orden en el que 
se programen los exámenes, el nivel de complejidad de las asignaturas.

\paragraph{Confección de los tribunales de tesis:}
Como parte del proceso de graduación cada año 
es necesaria la planificación de los tribunales de tesis para 
realizar las defensas de los trabajos de grado. Para la organización 
de esta tarea se necesita la coordinación de los profesores que formarán parte 
de los tribunales de acuerdo a los horarios asignados, además estos profesores 
deben tener conocimientos sobre los temas que se aborden en las tesis en que participen. \\


Este trabajo tiene como objetivo el desarrollo de una herramienta informática que permita
la integración de los procesos mencionados anteriormente, pero se centra en resolver
primeramente la asignación de docencia y  la confección de tribunales
de tesis. A continuación se describen ...


\subsection{Asignación de docencia}

En la facultad de Matemática y Computación de La Universidad de la 
Habana (MATCOM) se estudian las carreras de Matemática y Ciencia de la Computación. 
Cada carrera se rige por un plan de estudio, que es un programa estructurado
donde se definen los contenidos que forman parte del currículo de los 
graduados. En el plan de estudio de una carrera 
se debe caracterizar la profesión, describir las principales funciones de un profesional y 
los valores a desarrollar durante la carrera. 
Además determina la cantidad de horas clase de cada asignatura,
así como el año y semestre en el que se deben impartir.
Cada cierto tiempo el plan de estudio de una carrera universitaria cambia, principalmente
para mantener la formación de los profesionales actualizada con el desarrollo de la 
sociedad.

Actualmente en la carrera de Ciencia de la 
Computación de la Facultad de Matemática 
y Computación de La Universidad de La Habana, exiten tres planes de estudios 
vigentes. En el año 2018 se actualizó el plan de estudio para los profesionales de la 
carrera de Ciencia de la Computación, se creó el plan de estudio E,
donde se redujo la cantidad de años escolares de 5 a 4. Además, como consecuencia de la situación epidemiológica que provocó la COVID-19 
fue necesaria la modificación del plan E para la reducción de asignaturas del currículo, creándose 
el plan de estudio E (COVID). Por tanto los estudiantes que ingresaron en la facultad en 
el curso 2017-2018, que se encuentran en el quinto año de la carrera, se rigen por el plan D, 
los estudiantes que se encuentran en el primer y tercer año se rigen por el plan E y los 
estudiantes que cursan el segundo año se rigen por el E(COVID).   


% Se realiza una planificación de las distintas 
% disciplinas que se deben estudiar. Las disciplinas están compuestas por una o varias 
% asignaturas y en el plan de estudio se debe especificar la cantidad de horas clase de 
% cada asignatura, 

En MATCOM existen 4 departamentos: Matemática, Matemática Aplicada,
Programación y Sistemas de Información e Inteligencia 
Artificial y Sistemas Computacionales.
En un departamento se realizan actividades de coordinación de docencia, investigación y 
tutoría de estudiantes. Los departamentos están formados (voz pasiva) por profesores
y cuentan con un jefe de departamento. Los profesores que integran un departamento tienen 
un grado científico (Licenciado, Máster, Doctor) y una categoría docente (Titular, Auxiliar, Instructor, Asistente) .
Los departamentos de una 
de una facultad pueden brindan servicios a otras facultades, por ejemplo el departamento de 
Matemática Aplicada imparte en el segundo año de la carrera de Geografía la asignatura 
Estadística.


La dirección de la facultad realiza la planificación de la docencia, que consiste 
en determinar las asignaturas que se deben impartir en el semestre,
así como la cantidad de grupos de conferencias, clases prácticas u otras 
actividades de clase. Cada departamento recibe la planificación 
correspondiente a las asignaturas de las cuales se ocupa.
En la tabla \ref{tabla-planificación} se muestra un fragmento 
de la planificación de docencia para el departamento de Matemática 
Aplicada. La notación en la columna de los grupos n/m indica que exiten n 
cantidad de grupos de conferencia y m cantidad de grupos de clases 
prácticas.

\begin{table}[h!]
    \centering
    \begin{tabular}{| c | c | c | c | c |}
        \hline
        \thead{Facultad}   & \thead{Año} & \thead{Asignatura} & \thead{Horas} & \thead{Grupos}  \\ \hline
        MATCOM     & M3  & Optimización Matemática I  &  64   &  1/1   \\ 
        MATCOM     & C3  & Modelos de Optimización I  &  64   &  1/2   \\ 
        GEOGRAFÍA  & G2  & Estadística                &  80   &  1/2   \\ 
        \hline
    \end{tabular}
    \caption{Fragmento de la planificación de docencia.}
    \label{tabla-planificación}
\end{table}




La asignación de docencia en un departamento consiste en distribuir a los profesores
para cumplir con la planificación de docencia. En la tabla \ref{tabla-asignación} se muestra 
un fragmento de una posible asignación de docencia para la planificación de la 
tabla \ref{tabla-planificación}.

\begin{table}[H]
    \centering
    \begin{tabular}{ | c | c | c | c |}
      \hline
      \thead{Asignatura} & \thead{Horas} & \thead{Grupos} & \thead{Profesores}\\
      \hline
      Optimización Matemática I &  64  & 1/1 & \makecell{C: PT. Aymeeé Marrero (32) \\ CP: PT. Gemayqzel Bouza(32)} \\
      \hline
      Modelos de Optimización I   &  64   &  1/2 & \makecell{C: PT. Aymeeé Marrero(16) \\ C: Pas. Fernando Rodríguez (16) \\ CP: AD. Daniela González(32) \\ CP: INST. Camila Pérez(32)}    \\ 
      \hline
      Estadística                 &  80   &  1/2 &  \makecell{C: Elianys García (48) \\1CP: Ernesto Borrego(32) \\ 1CP: Niorlys o Elianys (32)} \\  
      \hline
    \end{tabular}
    \caption{Fragmento de la planificación de docencia.}
    \label{tabla-asignación}
\end{table}


[CONSULTAR QUE PROFESORES PONER EN LA TABLA DE ASIGNACION, y si poner 
la categoria docente en la tabla.]
Explicar que significa C y CP y las horas. \\


[EXPLICAR que hay asignaturas donde las conferencias para un mismo 
grupo se imparten por dos profesores distintos.] \\

Cada asignatura tiene un total de horas que se deben impartir de acuerdo 
al plan de estudio al que pertenezcan. La distribución del total de horas 
en las distintos tipos de clases como conferencias, clases prácticas, seminarios, 
laboratorios u otros, se maneja por el colectivo de profesores de la asignatura e incluso
puede cambiar de un año a otro. Por ejemplo, para la asignatura Modelos de Optimización I
que se rige por el plan de estudio D para la carrera de Ciencia de la Computación, que 
se imparte en el tercer año, se tiene un total de horas a impartir de 64,
de esas, 32 horas son de conferencia y 32 horas son de clases prácticas.






% El departamento de Matemática es responsable de la disciplina de Matemática Básica 
% que abarca las áreas: Álgebra, Análisis Matemático y Ecuaciones Diferenciales Ordinarias.

% El departamento de Matemática Aplicada abarca las áreas: Probabilidades y Estadística,
% Matemática Numérica y Optimización correspondientes a la disciplina de Matemática Aplicada.

% El departamento de Programación y Sistemas de Información.
% El departamento de Computación 2.

% que abarca las áreas del conocimiento 
% matemático 


% , Departamento de Matemática Aplicada abarca las áreas del conocimiento
% matemático 
% Aplicadas, 







% A continuación se describen los procesos de asignación de docencia y 
% confección de tribunales de tesis que son llevados
% a cabo en la Facultad de Matemática y Computación de La 
% Universidad de La Habana:


% El proceso de asignación de docencia en un departamento consiste en, 
% dadas las asignaturas que se imparten en un semestre y el conjunto de 
% profesores del departamento, realizar una distribución que satisfaga los
% requerimientos de dichas asignaturas.

% contemplando la experiencia y 
% las preferencias de los profesores.

% Las carreras universitarias que se estudian en La Universidad de La Habana están
% asociadas a un plan de estudio. El plan de estudio es un programa estructurado
% donde se definen los contenidos que forman parte del currículo de los 
% graduados. En el plan de estudio de una carrera 
% se debe caracterizar la profesión, describir las principales funciones de un profesional y 
% los valores a desarrollar durante la carrera. Se realiza una planificación de las distintas 
% disciplinas que se deben estudiar. Las disciplinas están compuestas por una o varias 
% asignaturas y en el plan de estudio se debe especificar la cantidad de horas clase de 
% cada asignaturas, así como el año y semestre en el que se deben impartir.

% Cada cierto tiempo el plan de estudio de una carrera universitaria cambia, principalmente
% para mantener la formación de los profesionales actualizada con el desarrollo de la 
% sociedad. Actualmente en la carrera de Ciencia de la 
% Computación de la Facultad de Matemática 
% y Computación de La Universidad de La Habana, exiten tres planes de estudios 
% vigentes. En el año 2018 se actualizó el plan de estudio para los profesionales de la 
% carrera de Ciencia de la Computación, se creó el plan de estudio E, donde el total de años 
% de estudio de la carrera se redujo de 5 a 4. Los estudiantes que ingresaron en la facultad en 
% el curso 2017-2018 se encuentran en el quinto año de la carrera y se rigen por el plan D. 
% Además, en consecuencia a la situación epidemiológica que provocó la epidemia de COVID-19 
% fue necesaria la modificación del plan E para la reducción de asignaturas del currículo, creándose 
% el plan de estudio E (COVID).   

% Las asignaturas que se imparten en una carrera universitaria se distribuyen 
% en los departamentos de la facultad, en ocasiones una facultad brinda servicio a otras facultades
% para .  


% TABLA EJEMPLO DE ASIGNACION DE DOCENCIA


% le llegan desde (direccion de la facultad) asignaturas y cantidad de grupos de conferencia y clase practicas 
% A los efectos del departamento se debe solo decidir que profesor 
% imparte cada asignatura. maneja la cant de profesores por act ect.




% La planificación de la asignación de docencia se lleva a cabo por el jefe del 
% departamento. Se debe tener en cuenta la categoría docente, grado científico y carga docente 
% de los profesores para la confección de la asignación. 



% sus particularidades, como la cantidad de horas
% a impartir en conferencias, clases prácticas, seminarios, laboratorios u
% otras modalidades, y la cantidad de grupos de clase que reciben dicha 
% actividad. Estas particularidades varían en dependencia de la matrícula 
% del curso y del plan de estudio al que la asignatura pertezca.


% además se tiene que las conferencias se imparten a un solo grupo mientras
% que las clases prácticas a dos.


\subsection{Tribunales de tesis}

El ejercicio de culminación de estudios de pregrado consiste en la 
realización de un trabajo de diploma o tesis.
La tesis es un trabajo de investigación que realizan
los estudiantes bajo la tutela de uno o varios profesores (tutores y cotutores)
sobre un tema que tribute a alguna de las áreas que abarca la carrera.
El ejercicio se compone de dos partes, la escritura de un documento que recoja
todo el proceso investigativo y posteriormente la exposición o defensa del mismo ante 
un tribunal.

El tribunal de tesis debe se conforma por n profesores, que asumen los roles de 
secretario, presidente y oponente. La cantidad de miembros de un tribunal y los roles 
que deben cumplir puede variar de un año a otro de acuerdo a las condiciones del curso 
escolar. Los profesores deben tener 
dominio acerca de los temas que se aborden en la investigación.
Este año los tribunales de tesis se formarán por n profesores con 
m roles.

El proceso de confección de los tribunales de tesis consiste en determinar cuáles 
profesores conformarán el tribunal de cada una de las tesis que se defienden cada año como parte 
del ejercicio de culminación de estudio. Una vez estén definidos (voz pasiva) los tribunales es necesario planificar la fecha, hora y 
lugar donde se efectuará la defensa de la tesis.


[Debo hablar de que ser oponente es una carga mucho mayor].
[Debo hablar sobre los principales problemas de realizar esa tarea 
de forma manual].  \\


% Para realizar la confección de un tribunal de tesis
% se necesitan definir tres incógnitas fundamentales: el local,
% la fecha y el tribunal. El tribunal debe estar compuesto por tres
% profesores que asuman los roles de secretario, presidente y 
% oponente. Los profesores que integren el tribunal no pueden ser 
% tutores o cotutores de la tesis en cuestión, y deben tener dominio
% sobre el tema que será presentado.
% Para la elección de la fecha y el local se debe tener en cuenta que la 
% defensa de una tesis dura aproximadamente 45 minutos.

% (No se si agregar algo sobre esto)
% lo mismo que en el proceso anterior, mencionar las desventajas de 
% hacer este proceso no informatizado

% La ejecución de los procesos se lleva a cabo en los distintos
% departamentos de la facultad y en su mayoría, si no en su totalidad,
% se realizan actualmente de manera manual y no informatizada.
% A continuación se describen algunos de estos procesos.



Para mantener toda la información relevante a los procesos de 
asignación de docencia y confección de los tribunales de tesis 
es imprescindible el uso de bases de datos.


\section{Diseño de bases de datos}
El desarrollo de la informática y la computación ha permitido
el almacenamiento de grandes cantidades de datos en espacios 
físicos limitados. Actualmente los sistemas de bases de datos juegan
un papel fundamental en el desarrollo de todo tipo de sistemas computacionales.

El diseño de base de datos es un proceso fundamental a la hora 
de modelar el conjunto de datos y las operaciones que se deseen
realizar sobre ellos. Un correcto diseño de la base de datos
es esencial para garantizar la consistencia de la información,
eliminar datos redundantes, ejecutar consultas de manera 
eficiente y mejorar el rendimiento de la base de datos [\cite{db_book_cap2}]. 

\subsection{Metodología de diseño de bases de datos}
La complejidad de la información y la cantidad de requisitos que se 
deseen modelar en un sistema computacional hacen que el diseño de una 
base de datos no sea una tarea sencilla. Por tanto es común es común 
descomponer el proceso de diseño en tres etapas fundamentales: diseño conceptual,
diseño lógico y diseño físico. 


\subsubsection{Diseño conceptual}
En esta fase se obtiene como resultado una representación 
abstracta y de alto nivel de la realidad a partir de las 
especificaciones de requisitos de usuario [REF libro]. El diseño
conceptual comienza con la identificación de las necesidades de los 
usuarios, que a menudo se pueden obtener a través de: examinar
la documentación existente como formularios, observando y
analizando el procesamiento de la información en el proceso que 
se desea modelar o mediante entrevistas a los usuarios finales [\cite{db_requirement_analysis}].

El objetivo del diseño conceptual es describir el contenido de 
información de la base de datos, mediante un esquema conceptual de
la base de datos [\cite{db_book_cap3}], que es independiente del SGBD
que se utilice para manipular la base de datos.
Los programadores o diseñadores utilizan modelos conceptuales de datos 
para la construcción de esquemas. El modelo entidad-relación es uno 
de los más utilizados para el diseño conceptual de bases de datos. Fue 
introducido por Peter Chen en 1976 [REF], y está formado por 
un conjunto de conceptos que permiten describir la realidad mediante 
representaciones gráficas y lingüísticas.

La metodología utilizada 
para el desarrollo del modelo conceptual de datos para la modelación 
de los procesos de asignación de docencia y confección de tribunales de 
tesis fue la metodología genérica MER/XX, que se basa en un enfoque
híbrido entre el modelo entidad/relacional extendido y los conceptos de la 
modelación orientada a objetos [\cite{db_book_cap2}]. 

El modelo entidad-relación extendido describe con un alto nivel de abstracción
el significado de los datos, las relaciones entre ellos y las reglas de negocio 
de un sistema de información. Sus dos elementos principales son las entidades y 
las relaciones, además existen extensiones al modelo básico 
como atributos de las entidades o cardinalidades de las relaciones, 
que aportan al modelo una mayor expresividad.



\subsubsection{Diseño lógico}
Es el proceso de transformar el esquema conceptual del dominio de la aplicación
que se obtiene en la fase anterior,
en un esquema para el modelo de datos subyacente a un SGBD particular.
Existen diferentes modelos matemáticos utilizados para el diseño lógico
de las bases de datos, tales como: el modelo de listas invertidas[REF], el modelo 
jerárquico[REF], el modelo de redes[REF] y el modelo relacional[REF]. 

El modelo relacional fue propuesto por Edgar Frank Codd en 1970. Es un 
modelo que se basa en la lógica de predicados y en la teoría 
de conjuntos, donde todos los datos se representan en términos de tuplas, 
agrupados en relaciones.

El modelo relacional fue el primer modelo de base de datos que se describió en términos 
matemáticos formales. A pesar de que existen  
implementaciones del modelo de base de datos relacional como lo definió originalmente 
Codd, no han tenido éxito popular hasta el momento. No obstante al modelo relacional se le 
atribuye un gran valor por su desarrollo teórico, que ha sido fundamento de muchos de los 
sistemas de gestión de bases de datos relacionales que se utilizan hoy en la actualidad, como:
MySQL, Oracle, SQL Server, PostgreSQL y Microsoft Access. 

El resultado de esta fase consiste en una descripción de las estructuras 
de datos utilizadas para almacenar la base de datos[\cite{db_book_cap3}], que se 
ajuste al modelo que utilice el SGBD. Esto quiere decir que,
si el modelo conceptual es expresado como un modelo 
entidad-relación y el modelo lógico utilizado en el SGBD es el modelo 
relacional, entonces las entidades, relaciones y los atributos del modelo entidad-relación
deben representarse como relaciones. 

NOTA: ACLARAR el concepto de relación en el modelo relacional


% El objetivo del diseño de bases de datos lógicas es crear 
% tablas bien estructuradas que reflejen adecuadamente el 
% entorno empresarial de la empresa.

\subsubsection{Diseño físico}
En esta etapa se transforma la estructura obtenida en la etapa del diseño
lógico, con el objetivo de conseguir mayor eficiencia. Es necesario
evaluar los aspectos de implementación física relacionados al SGBD que se utilice.
Por ejemplo: si se trabaja con una base de datos relacional, la 
transformación de la estructura puede consistir en crear una nueva relación que 
sea la combinación de varias relaciones o separar una relación en varias relaciones o 
añadir algún atributo calculable a una relación. \\




% Para el almacenamiento de los datos hizo uso de SQLite 3
% como sistema de gestión de bases de datos relacionales,
% el cual viene integrada en Django por defecto.

\subsection{Correcto diseño de bases de datos relacionales}

El diseño de una base de datos mediante el enfoque relacional, es una 
tarea no determinista, es posible obtener distintos esquemas relacionales 
como propuestas de la base de datos.
Un diseño incorrecto de una base de datos puede no responder apropiadamente a las exigencias
del proceso que se modela, y puede conllevar a la generación de dificultades o 
errores en el acceso a los datos. Entre los principales errores o dificultades que se pueden generar
se encuentran:

\begin{itemize}
    \item \textbf{Redundancia en los datos}, provoca un aumento innecesario 
    del tamaño de la base de datos, disminuyendo la eficiencia, además puede provocar 
    inconsistencia de los datos que puede conducir a la corrupción de los mismos. 
    \item \textbf{Violación de la integridad de los datos}, el término ``integridad de los datos'' se 
    refiere a la correctitud y completitud de la información en una base de datos. Cuando los datos son 
    modificados con sentencias INSERT, DELETE o UPDATE la integridad de los datos puede comprometerse.
\end{itemize}



Para lograr un correcto diseño de la base de datos se recomienda aplicar un método formal de 
análisis a cada uno de los esquemas obtenidos intuitivamente durante la fase de diseño conceptual, 
que se conoce como proceso de normalización.

\subsubsection{Normalización}
La normalización de una base de datos es un proceso que consiste 
en aplicar una serie de reglas a las relaciones obtenidas 
tras el paso del modelo entidad-relación (diseño conceptual) al modelo 
relacional (diseño lógico), con el objetivo de minimizar la redundancia de datos.
Cada regla se denomina una ``forma normal'', si se cumple la primera regla, se dice que 
la base de datos está en ``primera forma normal'', si se cumplen las tres 
primeras reglas se dice que la base de datos está en ``tercera forma normal''. El proceso de 
normalización se puede caracterizar como la transformación sucesiva de una colección de 
relaciones hacia una forma más restrictiva, de modo que mientrás mas profundo sea el nivel 
de normalización menor será la redundancia que albergue la base de datos [\cite{db_book_cap4}].
A continuación se introducen las definiciones de las reglas formales:

\begin{itemize}
    \item \textbf{Primera Forma Normal (1FN)}: una relación está en 1FN si y solo si, para 
    cada ocurrencia de la relación, toda tupla contiene exactamente un valor del dominio subyacente en
    cada atributo.
    \item \textbf{Segunda Forma Normal (2FN)}: una relación está en 2FN si, además de estar en 
    1FN, todos los atributos que no forman parte de alguna llave candidata constituyen información 
    acerca de la(s) llave(s) completa(s) y no de algún subconjunto de ella(s).
    \item \textbf{Tercera Forma Normal (3FN)}: una relación está en 3FN si, además de estar en 2FN,
    los atributos que no forman parte de alguna llave candidata constituyen información solo acerca
    de la(s) llave(s) y no acerca de otros atributos.
\end{itemize}


Aunque existen otros niveles de normalización como la ``Forma Normal de Boyce-Codd (BCFN)'',
la ``Cuarta Forma Normal(4FN)'' y la ``Quinta Forma Normal (5FN)'',
se considera que una base de datos en 3FN presenta niveles nulos o 
admisibles de redundancia en los datos[\cite{ws_3FN}].





% Para la implementación del sistema de gestión es necesario el almacenamiento
% y procesamiento de los datos que intervienen en los procesos de asignación de 
% docencia y confección de tribunales de tesis. A continuación se brinda una 
% descripción de cómo se modelaron estos procesos.





\section{Herramientas utilizadas para el desarrollo del software}
Para la informatización de los procesos
de asignación de docencia y confección de los tribunales de tesis
se implementa 
un sistema de gestión a través de una aplicación web,
utilizando un modelo cliente-servidor [REF].
Para el desarrollo en el lado del servidor se hace uso de Django [REF], en particular Django Rest Framework, aprovechando la
versatilidad, seguridad, escalabilidad y mantenibilidad
que ofrece.
El cliente se desarrolló usando Quasar [REF], creando una
\textit{single-page application} (SPA).
A continuación se profundiza en el modelo 
y herramientas utilizadas.

\subsection{Modelo cliente-servidor}
[VOLVER A REDACTAR ESTE PARRAFO]
El modelo cliente-servidor es uno de los más populares en
el mundo de la informática actual [REF], se utiliza en el 
desarrollo de todo tipo de aplicaciones, incluso muchos 
de los protocolos utilizados a diario implementan este modelo,
tales como File Transfer Protocol (FTP), Simple Mail Transfer Protocol
(SMTP) y Hypertext Transfer Protocol (HTTP).

El modelo cliente-servidor se define como una arquitectura
de software compuesta por los proveedores de un recurso o servicio, 
denominados servidores, y los solicitantes del servicio, denominados
clientes. En este modelo se realiza una comunicación de procesos
que implica el intercambio de datos tanto por parte del cliente 
como del servidor, realizando cada uno funciones diferentes. La comunicación
se realiza frecuentemente a través de una red informática, pero 
tanto el cliente como el servidor pueden residir en un mismo sistema.

Entre las principales ventajas que ofrece este modelo se encuentran:

\begin{itemize}
    \item centralización de los recursos, ya que los accesos y la integridad de los datos se controlan en el lado del servidor, de forma que un programa cliente defectuoso o no autorizado no puede dañar el sistema.
    \item división del procesamiento de la aplicación en varios dispositivos.
    \item permite la escalabilidad del sistema, en cualquier momento se puede mejorar la capacidad de procesamiento aumentando la cantidad de servidores sin que el funcionamiento de la red se vea afectado.
    \item la separación en cliente y servidor permite el uso de tecnologías enfocadas en las labores específicas que realiza cada uno. 
\end{itemize}
 

En la siguiente sección se describen las características de la tecnología 
que se utilizó en el desarrollo de la aplicación web, en el lado del servidor.

\subsection{Desarrollo del servidor}
Para la implementación de los servicios que consumen los clientes del 
sistema de gestión se hace uso de la biblioteca Django [REF]. Django permite el desarrollo de 
sitios web seguros y mantenibles [REF]. Entre sus principales características
se encuentran:

\begin{itemize}
    \item Sigue la filosofía
    de diseño DRY (Don't Repeat Yourself), brindando un conjunto de funcionalidades implementadas
    por los desarrolladores para evitar la repetición de código en procesos
    comunes en el desarrollo web.
    \item Implementa el patrón de diseño Model-View-Template (MVT), que
    consta de tres componentes esenciales Modelo, Vista,
    Plantilla. Estas componentes son responsables de diferentes
    partes del código e incluso pueden utililizarse de forma independiente.
    \item Está implementado en Python, por lo que cuenta con un 
    extenso conjunto de bibliotecas para resolver distintas tareas. Entre
    las bibliotecas más utilizadas resalta Django REST framework, desarrollada 
    para la creación de interfaces de programación de aplicaciones (APIs).
    [X q es relevante en este trabajo?]
    \item Provee un Object Relational Mapper (ORM), 
    que permite la interacción con la base de datos
    de forma orientada a objetos. Brinda la posibilidad de crear tablas, insertar,
    editar, borrar y extraer datos sin escribir consultas SQL, acelerando el proceso
    de desarrollo web. 
    \item Sigue también la filosofía de ``Batteries Included'' o ``Baterías Incluidas''
    proporcionando una biblioteca estándar rica y versátil con herramientas para
    crear sistemas complejos sin la necesidad de instalar paquetes separados. Algunas de 
    estas ``baterías'' son ``Django Admin'', ``Django ORM'', ``Authentication'' y
    ``HTTP''.
    \item Cuenta con extensa documentación, desde la documentación oficial hasta 
    contenido en forma de artículos, tutoriales y cursos en línea.
    \item Es escalable, utiliza una arquitectura de ``shared-nothing'' o ``nada compartido'',
    que permite agregar hardware en cualquier nivel: servidores de bases de datos,
    servidores de almacenamiento en caché o servidores web (REFERENCIA).

\end{itemize}

 
Aunque Django propone sus herramientas para el desarrollo del cliente con el uso 
de Plantillas, se decidió utilizar Quasar, los motivos se describen en la próxima sección.



\subsection{Desarrollo del cliente}
Dentro de las tecnologías de desarrollo web más utilizadas, 
JavaScript es una de las más populares en la actualidad [REF]. JavaScript es 
un lenguaje de programación basado en prototipos y multiparadigma.
Como JavaScript está implementado (voz pasiva) en la mayoría de los 
navegadores web, se utiliza para el desarrollo del cliente permitiendo el 
mejorar las interfaces de usarios y la creación de
páginas web dinámicas.
La popularidad de JavaScript para el desarrollo web ha impulsado
la creación de distintas bibliotecas y herramientas, entre las más 
utilizadas se encuentran: Angular, React y Vue [REF]. Angular es desarrollada 
y mantenida por Google, React por META (Facebook) y Vuejs por un 
equipo de colaboradores a nivel mundial, con Evan Yu, su principal
creador, un ex ingeniero de Google. 


% \subparagraph{Angular}
% Lanzado en 2010, Google está a cargo de desarrollar y 
% mantener Angular. La principal filosofía tras Angular es 
% tener todo lo que el desarrollador necesita, es un framework
% con muchas características integradas, tales como: validaciones 
% de formularios, envíos de solicitudes http, enrutamiento y 
% manejo de estados, cuenta además con herramientas adicionales
% como: línea de comandos, interfaces para el manejo y 
% creación de proyectos. Debido a esto muchos han llamado a Angular
% una plataforma, más que un framework. Google y Wix se 
% encuentran entre las empresas más populares que utilizan
% Angular.

% \subparagraph{React}
% Lanzado en 2013, Facebook desarrolla y mantiene React.
% A diferencia de Angular, React se enfoca en ser lo más minimalista posible,
% especializada en el desarrollo de interfaces de usuario, describiéndose
% a sí misma como una biblioteca. Cuenta con muchas menos funciones integradas
% que Angular por lo que se terminan utilizando muchos paquetes o
% dependencias de terceros al desarrollar un proyecto, ya sea para enrutamiento, manejo de estados o
% envío de solicitudes http.
% Whatsapp, Instagram, Paypal, Glassdoor y BBC son algunas de 
% las compañías populares que usan React.

% \subparagraph{Vue.js}
% Lanzado en 2014, su desarrollo y mantenimiento es llevado a 
% cabo por un equipo de colaboradores a nivel mundial, siendo  
% Evan Yu, su principal creador, un ex ingeniero de Google.
% Se puede decir que Vue se encuentra entre Angular y React en cuanto 
% a la cantidad de funciones integradas que ofrece, más que React, pero 
% menos que Angular. Sitios web como GitLab y Alibaba están usando Vue. \\


En los últimos años, Vue.js se ha convertido
en un fuerte competidor de Angular y React. El 
crecimiento de Vue.js se relaciona con su simplicidad 
y disponibilidad de bibliotecas y materiales de aprendizaje. A pesar de que Vue.js se define en su sitio oficial como ``An approachable, performant and versatile framework for building web user interfaces''
o ``Un marco accesible, eficaz y versátil para crear interfaces de usuario web''
no proporciona elementos o componentes de interfaz de usuarios. Es por esto que surgen  
otras bibliotecas encima de Vue.js, como una capa más de abstracción, que 
brindan un conjunto de componentes reutilizables y con estilos personalizables. Entre 
las más populares se encuentran: Vuetify, Bootstrap Vue y Quasar [REF].
Este último fue el escogido para el desarrollo del cliente de la aplicación web. 
  



\subsubsection{Quasar}
Es una biblioteca de código abierto desarrollado para Vue.js,
que permite el desarrollo aplicaciones/sitios web de una sola página (SPA),
aplicaciones/sitios web renderizados del lado del servidor (SSR),
aplicaciones web progresivas (PWA), incluso aplicaciones móviles y de 
escritorio, reutilizando el mismo código fuente. 
Entre los principales beneficios que Quasar ofrece 
se encuentran:

[ENFOCAR ESTO COMO X Q ES UTIL LO DE ABAJO PARA MI.]

\begin{itemize}
    \item más de 70 componentes web personalizables y de alto rendimiento para cubrir la 
    mayoría de las necesidades de la web.
    \item buenas prácticas integradas tales como: compresión de código, 
    manejo avanzado de caché, mapeo de fuentes y carga diferida. 
    \item Quasar-CLI, una herramienta que genera de forma automática la estructura del 
    proyecto en cuestión.
    % \item soporte para varios idiomas, incluyendo 
    % compatibilidad para idiomas RLT (right to left), tanto para los componentes de Quasar
    % como para el código del desarrollador.  
    \item amplia comunidad y documentación.
    % \item ofrece una versión UMD (Definición de módulo unificado) para la migraración
    % de proyectos existentes sin necesidad de ningún proceso de compilación.
    % \item promueve las buenas prácticas para mantener la seguridad de las aplicaciones
    % desarrolladas con Quasar, en la documentación oficial se le dedica un capítulo
    % a este tema [REF].

\end{itemize}


% La minificación es el proceso de reducir el peso de los archivos de código
% fuente a través de la eliminación de bytes innecesarios como los 
% saltos de línea, espacios adicionales y comentarios. Es uno de 
% los principales métodos utilizados para reducir los tiempos de carga y
% el uso de ancho de banda en los sitios web  (HTML/CSS/JS minification)





