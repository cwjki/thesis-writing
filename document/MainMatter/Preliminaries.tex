\chapter{Preliminares}\label{chapter:preliminaries}
Este capítulo provee una breve introducción a los distintos 
procesos que se llevan a cabo en un departamento de la Facultad
de Matemática y Computación de La Universidad de La Habana, así
como una descripción de las herramientas utilizadas en la 
solución brindada. 

\section{Funcionamiento de la facultad}
Cada año en la Facultad de Matemática y Computación de La 
Universidad de La Habana se realizan un conjunto de procesos de 
planificación y administración relacionados con los cursos escolares.
La ejecución de los procesos es delegada a los distintos
departamentos de la facultad y en su mayoría, si no en su totalidad,
se realizan actualmente de manera manual y no informatizada,
por ejemplo:

\begin{itemize}
    \item Asignación de docencia
    \item Confección de los tribunales de tesis 
    \item Asignación de cursos optativos
    \item Asignación de alumnos ayudantes
    \item Planificación de las evaluaciones del semestre
\end{itemize}


A continuación se describen los procesos de asignación de docencia y 
confección de tribunales de tesis que son llevados
a cabo en la Facultad de Matemática y Computación de La 
Universidad de La Habana:

\subsection{Asignación de docencia}
El proceso de asignación de docencia en un departamento consiste en, 
dadas las asignaturas a impartir en un semestre y el conjunto de 
profesores del mismo, realizar una distribución que satisfaga los
requerimientos de dichas asignaturas, contemplando la experiencia y 
las preferencias de los profesores.

Cada asignatura tiene sus particularidades, como la cantidad de horas
a impartir en conferencias, clases prácticas, seminarios, laboratorios u
otras modalidades, y la cantidad de grupos de clase que reciben dicha 
actividad. Estas particularidades varían en dependencia de la matrícula 
del curso y del plan de estudio al que la asignatura pertezca.

Por ejemplo para la asignatura Modelos de Optimización I, perteneciente
al plan de estudio D, para la carrera de Ciencia de la Computación que 
se imparte en el tercer año, se tiene un total de horas a impartir de 64,
de esas, 32 horas son de conferencia y 32 horas son de clases prácticas,
además se tiene que las conferencias se imparten a un solo grupo mientras
que las clases prácticas a dos.

(No se si agregar algo sobre esto)
Actualmente este proceso es ejecutado por una persona 'ciclano' y debe
ponerse de acuerdo con los profes, muchas veces.


\subsection{Tribunales de tesis}
Para realizar la confección de un tribunal de tesis
se necesitan definir tres incógnitas fundamentales: el local,
la fecha y el tribunal. El tribunal debe estar compuesto por tres
profesores que asuman los roles de secretario, presidente y 
oponente. Los profesores que integren el tribunal no pueden ser 
tutores o cotutores de la tesis en cuestión, y deben tener dominio
sobre el tema que será presentado.
Para la elección de la fecha y el local se debe tener en cuenta que la 
defensa de una tesis dura aproximadamente 45 minutos.

(No se si agregar algo sobre esto)
lo mismo que en el proceso anterior, mencionar las desventajas de 
hacer este proceso no informatizado




\section{Descripción del software}
Para la informatización de estos procesos se implementa 
un sistema de gestión a través de una aplicación web,
utilizando un modelo cliente-servidor.
Para el desarrollo en el lado del servidor se hace uso del framework Django, en particular Django Rest Framework, aprovechando la
versatilidad, seguridad, escalabilidad y mantenibilidad
que el framework ofrece.
Mientras que en el lado del cliente se hace uso
del framework Quasar, que se basa en Vue.js, creando una
single-page application (SPA) o aplicación de página única.
A continuación se profundiza en el modelo 
y herramientas utilizadas.

\subsection{Modelo cliente-servidor}
El modelo cliente-servidor es uno de los más populares en
el mundo de la informática actual, es utilizado para el 
desarrollo de todo tipo de aplicaciones, incluso muchos 
de los protocolos utilizados a diario implementan este modelo,
tales como File Transfer Protocol (FTP), Simple Mail Transfer Protocol
(SMTP) y Hypertext Transfer Protocol (HTTP).

El modelo cliente-servidor puede ser definido como una arquitectura
de software compuesta por los proveedores de un recurso o servicio, 
denominados servidores, y los solicitantes del servicio, denominados
clientes. En este modelo se realiza una comunicación de procesos
que implica el intercambio de datos tanto por parte del cliente 
como del servidor, realizando cada uno funciones diferentes. La comunicación
se realiza frecuentemente a través de una red informática, pero 
tanto el cliente como el servidor pueden residir en un mismo sistema.

Entre las principales ventajas que ofrece este modelo se encuentran:

\begin{itemize}
    \item centralización de los recursos, los accesos y la integridad de los datos son controlados por el servidor de forma que un programa cliente defectuoso o no autorizado no pueda dañar el sistema.
    \item divide el procesamiento de la aplicación en varios dispositivos.
    \item permite la escalabilidad del sistema, en cualquier momento se puede mejorar la capacidad de procesamiento
\end{itemize}
 


\subsection{Backend}
\subsection{Frontend}