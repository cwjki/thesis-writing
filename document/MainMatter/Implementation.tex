\chapter{Descripción las herramientas implementadas}\label{chapter:implementation}


\section{Salvar y llenar datos de la BD}

En el servidor se implementó un módulo para el trabajo con la base de datos que permite salvar el estado de la base de datos en documentos csv, de igual forma permite llenar la base de datos a partir de documentos csv.

Se crearon dos comandos para ejecutar en la terminal con el fin de realizar las tareas mencionadas previamente:

\subsection{save database}

El comando save database permite salvar tanto 
una tabla como la base de datos completa.

Para salvar solo una tabla se hace uso del 
argumento -m y luego el nombre de la 
tabla a salvar:

\begin{verbatim}
    python manage.py save_database -m Professors
\end{verbatim}


Para salvar la base de datos completa ejecutar 
solo el comando:

\begin{verbatim}
    python manage.py save_database
\end{verbatim}


\subsection{fill database}
El comando fill database permite llenar desde 
documentos csv tanto una tabla como la base de 
datos completa.

Análogamente al comando anterior para llenar solo una tabla se hace uso del argumento -m y luego el nombre de la tabla a llenar:

\begin{verbatim}
    python manage.py fill_database -m Subjects
\end{verbatim}

Para llenar la base de datos completa ejecutar 
solo el comando:

\begin{verbatim}
    python manage.py fill_database
\end{verbatim}

Los posibles nombres de entidades a utilizar con
el parámetro -m tanto para los comandos 
save database como fill database son:

ClassTypes, Faculties,
ScientificDegrees, TeachingCategories,
Semesters, TeachingGroups, TimePeriods,
Careers, StudyPlans, CarmenTable, Departments,
Subjects, Professors, SubjectDescriptions, TeachingAssignments,
Places, Keywords, Thesis, ThesisCommittee

\section{Generar csv}

Se implementó la funcionalidad de descargar las 
asignaciones de docencia y los tribunales de 
tesis en ficheros csv.

\section{Generar asignaciones de docencia}

Se implementó un modelo de optimización para la 
generación de asignaciones de docencia, pendiente
computar el peso de las asignaciones (dado las
preferencias de los profesores y sus habilidades)
