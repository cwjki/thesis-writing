\chapter{Descripción del diseño de la base de datos}\label{chapter:database}
El desarrollo de la informática y la computación ha permitido
el almacenamiento de grandes cantidades de datos en espacios 
físicos limitados. Actualmente los sistemas de bases de datos juegan
un papel fundamental en el desarrollo de todo tipo de sistemas computacionales.

El diseño de base de datos es un proceso fundamental a la hora 
de modelar el conjunto de datos y las operaciones que se deseen
realizar sobre ellos. Un correcto diseño de la base de datos
es esencial para garantizar la consistencia de la información,
eliminar datos redundantes, ejecutar consultas de manera 
eficiente y mejorar el rendimiento de la base de datos [REF LIBRO]. 

\subsection{Metodología de diseño de bases de datos}
La complejidad de la información y la cantidad de requisitos que se 
deseen modelar en un sistema computacional hacen que el diseño de una 
base de datos no sea una tarea sencilla. Por tanto es común es común 
descomponer el proceso de diseño en tres etapas fundamentales: diseño conceptual,
diseño lógico y diseño físico. 

\paragraph{Diseño conceptual:}
En esta fase se obtiene como resultado una representación 
abstracta y de alto nivel de la realidad a partir de las 
especificaciones de requisitos de usuario [REF libro]. Este 
proceso comienza con la identificación de las necesidades de los 
usuarios, que a menudo se pueden obtener a través de: examinar
la documentación existente como formularios [REF COBOL], observando y
analizando el procesamiento de la información en el proceso que 
se desea modelar o mediante entrevistas a los usuarios finales.
El objetivo del diseño conceptual es describir el contenido de 
información de la base de datos, mediante un esquema conceptual de
la base de datos [REF LIBRO U OTRO] que es independiente del SGBD
que se utilice para manipular la base de datos.
Los programadores o diseñadores utilizan modelos conceptuales de datos 
para la construcción de esquemas. El modelo entidad-relación es uno 
de los más utilizados para el diseño conceptual de bases de datos. Fue 
introducido por Peter Chen en 1976 [REF], y está formado por 
un conjunto de conceptos que permiten describir la realidad mediante 
representaciones gráficas y lingüísticas. La metodología utilizada 
para el desarrollo del modelo conceptual de datos para la modelación 
de los procesos de asignación de docencia y confección de tribunales de 
tesis fue la metodología genérica MER/XX [REF], que se basa en un enfoque
híbrido entre el modelo entidad/relacional extendido y los conceptos de la 
modelación orientada a objetos [REF LIBRO]. 


\paragraph{Diseño lógico:}
Es el proceso de transformar el esquema conceptual del dominio de la aplicación
que se obtiene en la fase anterior,
en un esquema para el modelo de datos subyacente a un SGBD particular.
Existen diferentes modelos matemáticos utilizados para el diseño lógico
de las bases de datos, tales como: el modelo de listas invertidas[REF], el modelo 
jerárquico[REF], el modelo de redes[REF] y el modelo relacional[REF]. 
El resultado de esta fase consiste en una descripción de las estructuras 
de datos utilizadas para almacenar la base de datos[REF LIBRO], que se 
ajuste al modelo que utilice el SGBD. Esto quiere decir que,
si el modelo conceptual es expresado como un modelo 
entidad-relación y el modelo lógico utilizado en el SGBD es el modelo 
relacional, entonces las entidades, relaciones y los atributos del modelo entidad-relación
deben representarse como relaciones[REF]. 


% El objetivo del diseño de bases de datos lógicas es crear 
% tablas bien estructuradas que reflejen adecuadamente el 
% entorno empresarial de la empresa.

\paragraph{Diseño físico:} A \\

Para la implementación del sistema de gestión es necesario el almacenamiento
y procesamiento de los datos que intervienen en los procesos de asignación de 
docencia y confección de tribunales de tesis. A continuación se brinda una 
descripción de cómo se modelaron estos procesos.

% Para el almacenamiento de los datos hizo uso de SQLite 3
% como sistema de gestión de bases de datos relacionales,
% el cual viene integrada en Django por defecto.

\section{Modelación de la asignación de docencia}
Para la modelación de este proceso se crearon las 
siguientes entidades:

Primeramente se modelaron las entidades básicas que intervienen 
en el proceso de la asignación de docencia. 




\subparagraph{Professor:}
Agrupa los datos asociados a los profesores, 
tales como: nombre, apellidos, categoría docente, grado 
científico y departamento al que pertenece.

\subparagraph{Subject:}
Agrupa los datos asociados a las asignaturas,
tales como: nombre, departamento responsable de impartir
la asignatura, plan de estudio, semestre en el que se 
imparte, cantidad de horas totales y carrera a la que 
pertenece.

\subparagraph{CarmenTable:}
Entidad creada con el fin de representar
el grupo escolar vigente en el año actual, cuenta con un
'TeachingGroup', semestre actual, período de tiempo y 
plan de estudio correspondiente. Por ejemplo representa 
que el grupo C3 (Computación tercer año) con el plan de 
estudio E, es el que esta vigente en el semestre actual.

\subparagraph{SubjectDescription:}
Entidad necesaria para separar las distintas actividades 
que se imparten en una asignatura ( conferencia, clases 
prácticas, seminarios, laboratorios, otras ) con el fin 
de poder realizar la asignación. Cuenta con una 'subject',
un tipo de actividad, y el CarmenTable correspondiente.


\subparagraph{TeachingAssignment:}
Entidad que modela la asignación de docencia, posee un
'Professor', una 'SubjectDescription' y
el porciento correspondiente a blah.

Los campos carrera, facultad, departamento, plan de 
estudio, grado científico, categoría docente, tipo de 
clase, semestre, período de tiempo y grupo escolar fueron
modelados como nomencladores en las entidades Career,
Faculty, Department, StudyPlan, ScientificDegree, 
TeachingCategory, ClassType, Semester, TimePeriod y 
TeachingGroup respectivamente.


\section{Modelación de los tribunales de tesis}
Para la modelación de este proceso se crearon las 
siguientes entidades:

\subparagraph{Thesis:}
Agrupa los datos asociados a una tesis, tales son:
título, estudiante, tutor, posibles cotutores y palabras
claves.

\subparagraph{ThesisCommittee:}
Entidad que representa un tribunal de tesis, cuenta con 
una 'Thesis', una fecha, un lugar y tres relaciones con 
la entidad 'Professor' para los roles de
secretario, presidente y oponente.

Los campos lugar y palabras claves fueron modelados como
nomencladores en las entidades 'Place' y 'Keyword'.