\chapter{Descripción del diseño de la base de datos}\label{chapter:database}
El desarrollo de la informática y la computación ha permitido
el almacenamiento de grandes cantidades de datos en espacios 
físicos limitados. Actualmente los sistemas de bases de datos juegan
un papel fundamental en el desarrollo de todo tipo de sistemas computacionales.
Un correcto diseño de la base de datos es esencial para garantizar 
la consistencia de la información, eliminar datos redundantes,
ejecutar consultas de manera eficiente y mejorar el 
rendimiento de la base de datos.

Para la implementación del sistema de gestión es necesario el almacenamiento
y procesamiento de los datos que intervienen en los procesos de asignación de 
docencia y confección de tribunales de tesis. A continuación se brinda una 
descripción de cómo se modelaron estos procesos.


% Para el almacenamiento de los datos hizo uso de SQLite 3
% como sistema de gestión de bases de datos relacionales,
% el cual viene integrada en Django por defecto.

\section{Modelación de la asignación de docencia}
Para la modelación de este proceso se crearon las 
siguientes entidades:

\subparagraph{Professor:}
Agrupa los datos asociados a los profesores, 
tales como: nombre, apellidos, categoría docente, grado 
científico y departamento al que pertenece.

\subparagraph{Subject:}
Agrupa los datos asociados a las asignaturas,
tales como: nombre, departamento responsable de impartir
la asignatura, plan de estudio, semestre en el que se 
imparte, cantidad de horas totales y carrera a la que 
pertenece.

\subparagraph{CarmenTable:}
Entidad creada con el fin de representar
el grupo escolar vigente en el año actual, cuenta con un
'TeachingGroup', semestre actual, período de tiempo y 
plan de estudio correspondiente. Por ejemplo representa 
que el grupo C3 (Computación tercer año) con el plan de 
estudio E, es el que esta vigente en el semestre actual.

\subparagraph{SubjectDescription:}
Entidad necesaria para separar las distintas actividades 
que se imparten en una asignatura ( conferencia, clases 
prácticas, seminarios, laboratorios, otras ) con el fin 
de poder realizar la asignación. Cuenta con una 'subject',
un tipo de actividad, y el CarmenTable correspondiente.


\subparagraph{TeachingAssignment:}
Entidad que modela la asignación de docencia, posee un
'Professor', una 'SubjectDescription' y
el porciento correspondiente a blah.

Los campos carrera, facultad, departamento, plan de 
estudio, grado científico, categoría docente, tipo de 
clase, semestre, período de tiempo y grupo escolar fueron
modelados como nomencladores en las entidades Career,
Faculty, Department, StudyPlan, ScientificDegree, 
TeachingCategory, ClassType, Semester, TimePeriod y 
TeachingGroup respectivamente.


\section{Modelación de los tribunales de tesis}
Para la modelación de este proceso se crearon las 
siguientes entidades:

\subparagraph{Thesis:}
Agrupa los datos asociados a una tesis, tales son:
título, estudiante, tutor, posibles cotutores y palabras
claves.

\subparagraph{ThesisCommittee:}
Entidad que representa un tribunal de tesis, cuenta con 
una 'Thesis', una fecha, un lugar y tres relaciones con 
la entidad 'Professor' para los roles de
secretario, presidente y oponente.

Los campos lugar y palabras claves fueron modelados como
nomencladores en las entidades 'Place' y 'Keyword'.