\chapter{Descripción de las funcionalidades que se desean}\label{chapter:features}
El objetivo principal de este trabajo es la creación de
una herramienta, sitio web, portal para informatizar
y automatizar distintos procesos que se llevan a cabo 
en un departamento.

\begin{itemize}
    \item informatización de los datos
    \item asignación de docencia
    \item confección de tribunales de tesis
    \item (mencionar todos los procesos que desea informatizar e automatizar el profe Fernando?)
    \item automatizar los procesos anteriores
\end{itemize}


La primera funcionalidad que se desea es la informatización
de los datos que intervienen en estos procesos.
Una vez se tengan los datos se desea realizar la 
asignación de docencia de un departamento, así como la
confección de los tribunales de tesis.

\section{Informatización de la datos}
Primeramente se hace necesaria la informatización 
de la información que interviene en los procesos 
previamente mencionados, por tanto se desea implementar
un mecanismo que permita ingresar datos en el sistema
de gestión y que estos sean almacenados en bases de
datos para su posterior uso.

\section{Asignación de docencia}
Una vez el sistema cuente con todos los datos que 
intervienen en la asignación de docencia, se desea 
implementar un mecanismo que permita la asignación
de profesores a tareas o actividades relacionadas
con las asignatura (conferencias, clases prácticas,
laboratorios, seminarios, entre otras).
Se quiere además la generación de un documento
CSV que agrupe esta información.


\section{Confección de los tribunales de tesis}
Se desea implementar la funcionalidad de creación
de tribunales de tesis, dada una tesis poder asignar
los profesores que conformarán el tribunal en los
roles de secretario, presidente y oponente, así como
un lugar y fecha para la defensa de la misma. De igual
forma se desea la generación de un documento CSV con
esta información.
