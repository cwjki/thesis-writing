\chapter{Descripción de las funcionalidades que se desean}\label{chapter:features}
El objetivo principal de este trabajo es la creación de una herramienta
que permita informatizar dos de los procesos que se realizan en la Facultad de Matemática y
Computación de la Universidad de La Habana: asignación de docencia y planificación de las tesis. 
Se desea además que la herramienta sea lo sufientemente 
extensible para que permita, en un futuro, la integración de otros procesos como los que se mencionan en la 
sección \ref{section:funcionamiento de la facultad}.

% El objetivo principal de este trabajo es la creación de
% una herramienta que permita la informatización de los procesos de
% asignación de docencia, confección de los tribunales de tesis y planificación de las defensas
% de las tesis, que se llevan a cabo en la Facultad de Matemática y Computación
% de la Universidad de La Habana.
% Se desea además que la herramienta permita la 
% posterior integración de los procesos que se abordan en la 
% sección \ref{section:funcionamiento de la facultad}.


En las siguientes secciones se describen las funcionalidades
que se desean incorporar para cada proceso.

% El objetivo principal de este trabajo es la creación de
% una herramienta que permita la informatización de los procesos
% de administración y planificación que se llevan a cabo en la Facultad de Matemática y Computación
% de La Universidad de La Habana. 
% Se desea implementar un sistema de 
% gestión para realizar los procesos de asignación de docencia y 
% confección de los tribunales de tesis, y que permita 
% la posterior integración de los procesos que se abordan en la 
% sección \ref{section:funcionamiento de la facultad}.












% La primera funcionalidad que se desea es la informatización
% de los datos que intervienen en estos procesos.
% Una vez se tengan los datos se desea realizar la 
% asignación de docencia de un departamento, así como la
% confección de los tribunales de tesis. A continuación 
% se describen estos procesos.


\section{Asignación de docencia}
Se desea implementar una mecanismo que permita realizar la asignación de docencia 
facilitando el acceso a toda la información que interviene en este proceso. A continuación 
se describen las principales funcionalidades que se desean:


\begin{itemize}
    \item Conocer las asignaturas que se deben impartir en un período de tiempo dado, por ejemplo enero-julio 2021.
    \item Conocer durante el proceso de asignación la carga docente que se la ha asignado a cada profesor.
    \item Exportar la asignación de docencia a un documento con un formato preestablecido, que 
    es el que se utiliza actualmente en la facultad.
\end{itemize}

%     \item Realizar la asignación de docencia en un departamento, mostrando solo los datos relevantes 
%      para el mismo. Por ejemplo, cuando se realice la asignación de docencia para el departamento de Matemática Aplicada, solo 
%      se mostrarán las asignaturas y profesores del mismo.
%     \item Realizar filtrados o búsquedas a partir de los datos que intervienen en la asignación 
%           de docencia. Por ejemplo, saber las asignaturas que debe impartir un profesor en un semestre dado.


A continuación se describen como se desean utilizar las funcionalidades anteriores durante el 
proceso de asignación de docencia.

Al comienzo de un período de tiempo, el jefe de departamento debe realizar la planificación de docencia
del semestre. Se desea que la herramienta muestre las asignaturas que se deben impartir por ese departamento en el período de tiempo dado.
Por ejemplo en el período enero-julio del curso 2022, el departamento de Matemática Aplicada debe impartir las siguientes asignaturas.


\begin{table}[H]
    \centering
    \begin{tabular}{| c | c | c | c | c |}
        \hline
        \thead{Facultad}   & \thead{Año} & \thead{Asignatura} & \thead{Horas} & \thead{Grupos}  \\ \hline
        MATCOM     & M3  & Optimización Matemática I  &  64   &  1/1   \\ 
        MATCOM     & C3  & Modelos de Optimización I  &  64   &  1/2   \\ 
        GEOGRAFÍA  & G2  & Estadística                &  80   &  1/2   \\ 
        \hline
    \end{tabular}
    \caption{Fragmento de la planificación de docencia.}
    \label{tabla-planificación-cap2}
\end{table}

La primera fila de tabla indica que la asignatura Optimización Matemática I, 
se imparte en la facultad de MATCOM a los estudiantes de tercer año de la carrera de 
Matemática, con un total de horas clase de 64 y que es un solo grupo de conferencias y 
un solo grupo de clases prácticas. 

A partir de esta información, el jefe de departamento comienza el proceso de asignación.
Se desea que el sistema muestre los profesores que están disponibles para 
cubrir las actividades de clase de las asignaturas (conferencias, clases prácticas).
En el curso 2022, se le asignaron las 32 horas de conferencia de la asignatura Optimización Matemática I,
a la profesora Aymeeé Marrero y las 32 horas de clases prácticas a la profesora Gemayqzel Bouza.

Cada vez que el jefe de departamento realice una 
asignación, se desea que se actualice la carga docente que tendrá en el semestre el profesor asignado, de forma que se conozca la carga 
de todos los profesores durante el proceso de asignación. En la tabla \ref{tabla-asignación-cap2}
se muestra la asignación de docencias correspondiente a la planificación que se muestra en la 
tabla \ref{tabla-planificación-cap2}

\begin{table}[H]
    \centering
    \begin{tabular}{ | c | c | c | c |}
      \hline
      \thead{Asignatura} & \thead{Horas} & \thead{Grupos} & \thead{Profesores}\\
      \hline
      Optimización Matemática I &  64  & 1/1 & \makecell{C: Aymeeé Marrero (32) \\ CP: Gemayqzel Bouza(32)} \\
      \hline
      Modelos de Optimización I   &  64   &  1/2 & \makecell{C: Aymeeé Marrero(16) \\ C: Fernando Rodríguez (16) \\ CP: Daniela González(32) \\ CP: Camila Pérez(32)}    \\ 
      \hline
      Estadística                 &  80   &  1/2 &  \makecell{C: Elianys García (48) \\ CP: Ernesto Borrego(32) \\ CP: Elianys García(32)} \\  
      \hline
    \end{tabular}
    \caption{Fragmento de la asignación de docencia.}
    \label{tabla-asignación-cap2}
\end{table}

Para la asignación que se muestra en la tabla \ref{tabla-asignación-cap2}, 
el sistema de gestión debe mostrar las siguientes cargas docentes:

\begin{multicols}{2}
    \begin{itemize}
        \item Aymeeé Marrero: 48 horas
        \item Gemayqzel Bouza: 32 horas 
        \item Fernando Rodríguez: 16 horas 
        \item Daniela González: 32 horas 
        \item Camila Pérez: 32 horas
        \item Elianys García: 80 horas
        \item Ernesto Borrego: 32 horas 
    \end{itemize}
\end{multicols}

En este caso se puede apreciar que la carga docente de la profesora 
Elianys García es mayor que la del resto de los profesores, lo cual se 
quisiera evitar con la implementación de esta funcionalidad. 


Además se quiere que el sistema muestre las asignaturas o 
grupos que faltan por cubrir para cumplir con la planificación de docencia. 
Una vez concluida la asignación, se desea poder exportar la información de la docencia a un documento con el 
formato que se utiliza actualmente en la facultad. 







En la próxima sección se describen las funcionalidades que se desean incorporar para el 
proceso de planificación de las tesis.


\section{Planificación de las tesis}
El proceso de planificación de las tesis se puede descomponer en dos subprocesos principales: 
confección de los tribunales de tesis y planificación de las defensas de las mismas.
Para la planificación de las tesis, se desea disponer de las siguientes funcionalidades.



\begin{itemize}
    \item Cada vez que se agregue una tesis en el sistema, crear un tribunal para la misma.
    \item Conocer, durante el proceso de confección de los tribunales de tesis, la cantidad de tribunales 
    en los que participa cada profesor.
    \item Una vez se confeccione el tribunal de una tesis, crear un horario para la defensa de la misma.
    \item Una vez estén definidos los horarios, se desea conocer en cuáles, un determinado profesor no está libre (por participaciones en defensas). 
    Con el objetivo de facilitar cambios necesarios en la planificación de las defensas de tesis.  
    \item Exportar la confección de los tribunales de tesis a un documento con un formato preestablecido.
    \item Exportar la planificación de las defensas de tesis a un documento con un formato preestablecido.
\end{itemize}

% A continuación se describen como se desean utilizar las funcionalidades anteriores
% durante el proceso de planificación de tesis.

Para la ejemplificación de las funcionalidades que se desean durante el proceso de 
planificación de las tesis se utilizará un ejemplo.
En la tabla \ref{tabla-tesis-cap2}, se muestran dos de las tesis que se deben evaluar como 
parte del ejercicio de culminación de estudios del curso 2022.

\begin{table}[H]
    \centering
    \begin{tabular}{ | c | c | c | c |}
      \hline
      \thead{ID} & \thead{Tesis} & \thead{Estudiante} & \thead{Tutores} \\
      \hline 
             1 & \makecell{Simulación y optimización \\ de movimientos de malabares} & Gustavo Despaigne & Fernando Rodríguez  \\
      \hline
             2 & \makecell{Propagación de epidemias \\ mediante modelos basados \\ en metapoblaciones} & Abel Antonio Cruz & \makecell{Angela M. León \\ José A. Mesejo \\ Camila Pérez} \\
      \hline
    \end{tabular}
    \caption{Fragmento de dos de las tesis correspondientes al curso 2022}
    \label{tabla-tesis-cap2}
\end{table}

El encargado de realizar la planificación de las tesis debe confeccionar un tribunal para 
cada una de ellas. Se desea que el sistema de gestión muestre los profesores de la facultad para 
la confección de los tribunales. En el curso 2022 los tribunales estarán compuestos por los tutores, 
un oponente y un presidente, por tanto solo se deben determinar los profesores que asumirán los roles 
de oponente y presidente. En la figura tal se muestran posibles tribunales para las tesis correspondientes 
a la tabla \ref{tabla-tesis-cap2}.



\begin{table}[H]
    \centering
    \begin{tabular}{ | c | c | c | c |}
      \hline
      \thead{ID Tesis} & \thead{Tutores} & \thead{Oponente} & \thead{Presidente} \\
      \hline 
             1 & Fernando Rodríguez & Gemayqzel Bouza & Aymeeé Marrero  \\
      \hline
             2 & \makecell{Angela M. León \\ José A. Mesejo \\ Camila Pérez} & Damian Valdés & Gemayqzel Bouza  \\
      \hline
    \end{tabular}
    \caption{Posibles tribunales de tesis}
    \label{tabla-tribunal-tesis-cap2}
\end{table}

Con el objetivo de realizar una distribución equitativa de los profesores en los tribunales de tesis, 
se desea que durante el proceso de confección de los tribunales, se pueda conocer la cantidad en los que participa cada profesor.
En la tabla \ref{tabla-carga-profesores-tribunales}
se muestran las participaciones de los profesores en tribunales 
de acuerdo a la tabla \ref{tabla-tribunal-tesis-cap2}

\begin{table}[H]
    \centering
    \begin{tabular}{ | c | c | c |}
      \hline
      \thead{Profesor} & \thead{Oponente} & \thead{Presidente} \\
      \hline 
             Gemayqzel Bouza & 1 & 1  \\
      \hline
             Aymeeé Marrero & 0 & 1 \\
      \hline
            Damian Valdés & 1 & 0 \\
      \hline
    \end{tabular}
    \caption{Cantidad de tribunales de tesis en los que participa cada profesor}
    \label{tabla-carga-profesores-tribunales}
\end{table}


Una vez se definan los tribunales de tesis el próximo paso es planificar un horario
para la defensa de la misma. Se debe definir la fecha, hora y local en donde se realizará
este ejercicio. En la tabla tal se muestran posibles planificaciones para las defensas de las 
tesis que se muestran en la tabla \ref{tabla-tesis-cap2}.

\begin{table}[H]
    \centering
    \begin{tabular}{ | c | c | c | c |}
      \hline
      \thead{ID Tesis} & \thead{Fecha} & \thead{Hora} & \thead{Local} \\
      \hline 
             1 & 10/12/2022 & 2:00 PM & Aula Posgrado  \\
      \hline
             2 & 10/12/2022 & 3:30 PM & Salón del Decanato \\
      \hline
    \end{tabular}
    \caption{Posible horarios para las defensas de tesis}
    \label{tabla-defensa-tesis-cap2}
\end{table}

Una vez concluido el proceso de planificación de las tesis se desea poder exportar 
los datos de los tribunales y las planificaciones de defensas a documentos con un formato 
preestablecido.

Con el objetivo de evitar que una tesis se quede sin planificar, se desea que cada vez que una tesis se agregue, el sistema genere automáticamente 
un tribunal vacío y una planificación para la defensa de la misma, sin definir la fecha, hora y local.

Además se desea conocer para cada profesor los horarios que tiene ocupado en participaciones en tribunales
de tesis, por si es necesario realizar algún cambio en las planificaciones de las defensas.



% \item Realizar filtrados o búsquedas a partir de los datos que intervienen en la confección 
%           de tribunales de tesis. Por ejemplo, saber en cuáles tribunales participa un profesor y con qué rol (secretario, oponente, presidente). 
% \item Realizar filtrados o búsquedas a partir de los datos que intervienen en la planificación 
%     de las defensas de tesis. Por ejemplo, saber dado una fecha cuáles tesis se defienden en la misma o el local 
%     en donde se llevará a cabo el ejercicio.

En el próximo capítulo se describe como se modelaron los datos para realizar la asignación de 
docencia y la planificación de las tesis.


