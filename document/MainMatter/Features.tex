\chapter{Descripción de las funcionalidades que se desean}\label{chapter:features}
El objetivo principal de este trabajo es la creación de una herramienta
que permita informatizar dos de los procesos que se realizan en la Facultad de Matemática y
Computación de la Universidad de La Habana: asignación de docencia y planificación de las tesis. Se desea además que la herramienta sea lo sufientemente 
extensible para que permita la integración de otros procesos como los que se mencionan en la 
sección \ref{section:funcionamiento de la facultad}.

% El objetivo principal de este trabajo es la creación de
% una herramienta que permita la informatización de los procesos de
% asignación de docencia, confección de los tribunales de tesis y planificación de las defensas
% de las tesis, que se llevan a cabo en la Facultad de Matemática y Computación
% de la Universidad de La Habana.
% Se desea además que la herramienta permita la 
% posterior integración de los procesos que se abordan en la 
% sección \ref{section:funcionamiento de la facultad}.


En las siguientes secciones se describen las funcionalidades
que se desean incorporar para cada proceso.

% El objetivo principal de este trabajo es la creación de
% una herramienta que permita la informatización de los procesos
% de administración y planificación que se llevan a cabo en la Facultad de Matemática y Computación
% de La Universidad de La Habana. 
% Se desea implementar un sistema de 
% gestión para realizar los procesos de asignación de docencia y 
% confección de los tribunales de tesis, y que permita 
% la posterior integración de los procesos que se abordan en la 
% sección \ref{section:funcionamiento de la facultad}.












% La primera funcionalidad que se desea es la informatización
% de los datos que intervienen en estos procesos.
% Una vez se tengan los datos se desea realizar la 
% asignación de docencia de un departamento, así como la
% confección de los tribunales de tesis. A continuación 
% se describen estos procesos.


\section{Asignación de docencia}
Se desea implementar una mecanismo que permita realizar la asignación de docencia 
facilitando el acceso a toda la información que interviene en este proceso. A continuación 
se describen las principales funcionalidades que se desean:


\begin{itemize}
    \item Conocer las asignaturas que se deben impartir en un período de tiempo dado, por ejemplo enero-julio 2021.
    \item Conocer durante el proceso de asignación la carga docente que se la ha asignado a cada profesor.
    \item Exportar la asignación de docencia a un documento.
\end{itemize}

%     \item Realizar la asignación de docencia en un departamento, mostrando solo los datos relevantes 
%      para el mismo. Por ejemplo, cuando se realice la asignación de docencia para el departamento de Matemática Aplicada, solo 
%      se mostrarán las asignaturas y profesores del mismo.
%     \item Realizar filtrados o búsquedas a partir de los datos que intervienen en la asignación 
%           de docencia. Por ejemplo, saber las asignaturas que debe impartir un profesor en un semestre dado.



Al comienzo de un período de tiempo, el jefe de departamento debe realizar la planificación de docencia
del semestre. Se desea que la herramienta muestre las asignaturas que se deben impartir por ese departamento en el período de tiempo dado.
A partir de esta información, el jefe de departamento comienza el proceso de asignación. Cada vez que el jefe de departamento realice una 
asignación, se actualiza la carga docente que tendrá en el semestre el profesor asignado, de forma que se conoce la carga 
docente de todos los profesores durante el proceso de asignación. Además se quiere que el sistema muestre las asignaturas o 
grupos que faltan por cubrir para cumplir con la planificación de docencia. Una vez concluida la asignación, se desea poder
exportar la información de la docencia a un documento.

En la próxima sección se describen las funcionalidades que se desean incorporar para el 
proceso de planificación de las tesis.


\section{Planificación de las tesis}
El proceso de planificación de las tesis se puede descomponer en dos subprocesos principales: 
confección de los tribunales de tesis y planificación de las defensas de las mismas.
Para ello se desea disponer de las siguientes funcionalidades.



\begin{itemize}
    \item Cada vez que se agregue una tesis en el sistema, crear un tribunal y una planificación de la defensa para ella.
    \item Conocer durante el proceso de confección de los tribunales de tesis, la cantidad de tribunales 
    en los que participa cada profesor.
    \item Conocer los espacios de tiempo libre para cada profesor, por si es necesario realizar cambios de fechas para las defensas de tesis. 
    \item Exportar la confección de los tribunales de tesis a un documento.
    \item Exportar la planificación de las defensas de tesis a un documento.
\end{itemize}

El proceso de planificación de las tesis comienza con la agregación de las mismas en el sistema. 
Para evitar que una tesis se quede sin planificación, se desea que cada vez que una tesis se agregue, el sistema genere automáticamente 
un tribunal vacío y una planificación para la defensa de la misma, sin definir la fecha, hora y lugar.
El encargado de realizar este proceso primeramente debe asignar los profesores que formarán parte de los tribunales de 
cada tesis y luego definir la fecha, hora y lugar en la que se llevará a cabo la defensa. 
Con el objetivo de realizar una distribución equitativa de los profesores en los tribunales de tesis, 
se desea que durante el proceso de confección de los tribunales, se pueda conocer la cantidad en los que participa cada profesor.
Además se desea poder exportar los datos de los tribunales y de las defensas de las tesis a un documento. 

% \item Realizar filtrados o búsquedas a partir de los datos que intervienen en la confección 
%           de tribunales de tesis. Por ejemplo, saber en cuáles tribunales participa un profesor y con qué rol (secretario, oponente, presidente). 
% \item Realizar filtrados o búsquedas a partir de los datos que intervienen en la planificación 
%     de las defensas de tesis. Por ejemplo, saber dado una fecha cuáles tesis se defienden en la misma o el local 
%     en donde se llevará a cabo el ejercicio.

En el próximo capítulo se describe como se modelaron los datos para realizar la asignación de 
docencia y la planificación de las tesis.


