\chapter{Descripción de las funcionalidades que se desean}\label{chapter:features}
El objetivo principal de este trabajo es la creación de
una herramienta que permita la informatización de los procesos
de asignación de docencia y confección de los tribunales de tesis que se llevan a cabo en la Facultad de Matemática y Computación
de La Universidad de La Habana. Se desea que la  además permita la integración de otros procesos 

contemple la integración 

pero que permita la integración de otros procesos 

, sitio web o portal para informatizar
y automatizar distintos procesos que se llevan a cabo 
en un departamento, tales como:

\begin{itemize}
    \item asignación de docencia
    \item confección de tribunales de tesis
    \item (mencionar todos los procesos que desea informatizar y automatizar el profe Fernando?)
    \item automatizar los procesos anteriores
\end{itemize}


Para el desarrollo de un sistema de gestión que agrupe estos 
procesos se debe partir de la informatización de los datos necesarios 
para la ejecución de estas tareas. Por tanto se hace necesario 
implementar un mecanismo que permita ingresar, modificar y eliminar
la información. 


% La primera funcionalidad que se desea es la informatización
% de los datos que intervienen en estos procesos.
% Una vez se tengan los datos se desea realizar la 
% asignación de docencia de un departamento, así como la
% confección de los tribunales de tesis. A continuación 
% se describen estos procesos.

\section{Informatización de la datos}




% Se hace necesaria la informatización 
% de la datos que interviene en los procesos 
% previamente mencionados, por tanto se desea implementar
% un mecanismo que permita ingresar datos en el sistema
% de gestión y que estos sean almacenados en bases de
% datos para su posterior uso.

\section{Asignación de docencia}
Una vez el sistema cuente con todos los datos que 
intervienen en la asignación de docencia, se desea 
implementar un mecanismo que permita la asignación
de profesores a tareas o actividades relacionadas
con las asignatura (conferencias, clases prácticas,
laboratorios, seminarios, entre otras).
Se quiere además la generación de un documento
CSV que agrupe esta información.


\section{Confección de los tribunales de tesis}
Se desea implementar la funcionalidad de creación
de tribunales de tesis, dada una tesis poder asignar
los profesores que conformarán el tribunal en los
roles de secretario, presidente y oponente, así como
un lugar y fecha para la defensa de la misma. De igual
forma se desea la generación de un documento CSV con
esta información.
