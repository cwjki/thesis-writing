\chapter{Descripción de las funcionalidades que se desean}\label{chapter:features}
El objetivo de este trabajo es la creación de una herramienta
que permita informatizar dos de los procesos que se realizan en la Facultad de Matemática y
Computación de la Universidad de La Habana (MATCOM): la asignación de docencia y la planificación de las tesis. 

En este capítulo se describen las funcionalidades que se desean incorporar
en el sistema para cada proceso. 
Para ello, se ilustra el flujo de trabajo actual para 
realizar la asignación de docencia y la planificación de las tesis, resaltando 
cómo y cuándo utilizar las funcionalidades deseadas.
La sección \ref{docencia:cap2} está destinada a la asignación de docencia y 
la sección \ref{tesis:cap2} a la planificación de las tesis.



% el uso de las funcionalidades
% se describe el flujo de trabajo actual para realizar estos procesos y 


% En la sección \ref{docencia:cap2} 
% se ilustra el flujo de trabajo actual para realizar la asignación de docencia y cómo se desean utilizar 
% las funcionalidades y en la sección \ref{tesis:cap2}


% En las siguientes secciones se describen las funcionalidades
% que se desean incorporar para cada proceso. En la sección \ref{docencia:cap2}
% las relacionadas (voz pasiva) con la asignación de docencia y en la sección \ref{tesis:cap2}
% las relacionadas(voz pasiva) con la planificación de las tesis.

% Se desea además que la herramienta sea lo sufientemente 
% extensible para que permita, en un futuro, la integración de otros procesos como: 
% planificación de los exámenes finales y revalorizaciones, planificación de los alumnos ayudantes,
% planificación de los cursos optativos, entre otros.

% El objetivo principal de este trabajo es la creación de
% una herramienta que permita la informatización de los procesos de
% asignación de docencia, confección de los tribunales de tesis y planificación de las defensas
% de las tesis, que se llevan a cabo en la Facultad de Matemática y Computación
% de la Universidad de La Habana.
% Se desea además que la herramienta permita la 
% posterior integración de los procesos que se abordan en la 
% sección \ref{section:funcionamiento de la facultad}.



% El objetivo principal de este trabajo es la creación de
% una herramienta que permita la informatización de los procesos
% de administración y planificación que se llevan a cabo en la Facultad de Matemática y Computación
% de La Universidad de La Habana. 
% Se desea implementar un sistema de 
% gestión para realizar los procesos de asignación de docencia y 
% confección de los tribunales de tesis, y que permita 
% la posterior integración de los procesos que se abordan en la 
% sección \ref{section:funcionamiento de la facultad}.












% La primera funcionalidad que se desea es la informatización
% de los datos que intervienen en estos procesos.
% Una vez se tengan los datos se desea realizar la 
% asignación de docencia de un departamento, así como la
% confección de los tribunales de tesis. A continuación 
% se describen estos procesos.


\section{Asignación de docencia}\label{docencia:cap2}
% Se desea implementar un mecanismo que permita realizar la asignación de docencia 
% facilitando el acceso a toda la información que interviene en este proceso. 
% Las principales funcionalidades que se desean:

Como parte del proceso de informatización de la asignación docente, se 
desea incorporar las siguientes funcionalidades al sistema:

% Existen un conjunto de funcionalidades que se desean incorporar para realizar la asignación de 
% docencia, en el sistema que se propone en este trabajo, como:

\begin{itemize}
    \item Conocer las asignaturas que se deben impartir en un período de tiempo dado, por ejemplo enero-julio 2021.
    \item Conocer durante el proceso de asignación, la carga docente que se le ha asignado a cada profesor.
    \item Exportar la asignación de docencia a un documento con un formato preestablecido, que 
    es el que se utiliza actualmente en la facultad.
\end{itemize}

%     \item Realizar la asignación de docencia en un departamento, mostrando solo los datos relevantes 
%      para el mismo. Por ejemplo, cuando se realice la asignación de docencia para el departamento de Matemática Aplicada, solo 
%      se mostrarán las asignaturas y profesores del mismo.
%     \item Realizar filtrados o búsquedas a partir de los datos que intervienen en la asignación 
%           de docencia. Por ejemplo, saber las asignaturas que debe impartir un profesor en un semestre dado.


% A continuación se describen como se desean utilizar las funcionalidades anteriores durante el 
% proceso de asignación de docencia.


El siguiente ejemplo ilustra 
el flujo del proceso de asignación de docencia y cómo se 
desean utilizar las funcionalidades anteriores.
Supongamos que se desea realizar la asignación docente del departamento de 
Matemática Aplicada en el período enero--julio del curso 2022. A continuación 
se muestran las asignaturas que se deben impartir.

\begin{table}[H]
    \centering
    \begin{tabular}{| c | c | c | c | c |}
        \hline
        \thead{Asignatura}   & \thead{Año} & \thead{Facultad} & \thead{Horas} \\ \hline
        Programación y Algoritmos & M1 & MATCOM     & 64 \\
        \hline
        Seminarios de Problemas   & M1 & MATCOM     & 48 \\
        \hline
        Programación y Algoritmos & M2 & MATCOM     & 32 \\
        \hline
        Inferencia Estadística    & M3 & MATCOM     & 80 \\
        \hline
        Matemática Numérica       & M3 & MATCOM     & 48 \\
        \hline
        Optimización Matemática I & M3 & MATCOM     & 64 \\
        \hline
        Estadística               & C3 & MATCOM     & 64 \\
        \hline
        Modelos de Optimización I & C3 & MATCOM     & 64 \\
        \hline
        Estadística               & G2 & GEOGRAFÍA  & 80 \\
        \hline
        Estadística               & S2 & SOCIOLOGÍA & 80 \\
        \hline
    \end{tabular}
    \caption{Asignaturas que debe impartir el departamento de Matemática Aplicada en el período enero--julio 2022.}
    \label{tabla-asignaturas-cap2}
\end{table}

En la tabla \ref{tabla-asignaturas-cap2} se muestra los datos asociados 
a las asignaturas, como el nombre, el año escolar en el que se imparte, la 
facultad a la que pertenece el año escolar y la cantidad de horas totales. 

El primer requerimiento para el sistema, es que la herramienta muestren las asignaturas que se deben impartir por un 
departamento en un período de tiempo dado, por lo tanto, en este caso, se deberían mostrar todas las asignaturas que aparencen 
en la tabla \ref{tabla-asignaturas-cap2}. 

Para simplificar la descripción del proceso, vamos a asumir que las únicas 
asignaturas que se deben impartir son Optimización Matemática I, Modelos de 
Optimización I y Estadística (GEOGRAFÍA).

A partir de las asignaturas que se deben impartir en el semestre, se debe determinar la cantidad
de grupos de conferencias, clases prácticas o cualquier otra actividad de clase,
así como la distribución del total de horas a impartir en cada una de estas actividades. 
Esta tarea se referencia en el documento como ``planificar la carga de las 
asignaturas''.
En la tabla \ref{tabla-carga-asignatura-cap2} se muestra la planificación de la carga 
de las asignaturas que se deben impartir en este ejemplo.
    

% Al comienzo de un período de tiempo, el jefe de departamento debe realizar la planificación de docencia
% del semestre. Se desea que la herramienta muestre las asignaturas que se deben impartir por ese departamento en el período de tiempo dado.
% Por ejemplo en el período enero-julio del curso 2022, el departamento de Matemática Aplicada debe impartir las siguientes asignaturas.


\begin{table}[H]
    \centering
    \begin{tabular}{| c | c | c | c | c | c |}
        \hline
        \thead{Asignatura}  & \thead{Año} & \thead{Facultad} & \thead{Horas} & \thead{Grupos} & \thead{\makecell{Distribución \\ de horas}  } \\ \hline
        Optimización Matemática I  & M3  & MATCOM  &  64   &  1/1  & 32/32  \\ 
        Modelos de Optimización I  & C3  & MATCOM  &  64   &  1/2  & 32/32  \\ 
        Estadística                & G2  & GEOGRAFÍA  &  80   &  1/2  & 48/32  \\ 
        \hline
    \end{tabular}
    \caption{Carga de las asignaturas}
    \label{tabla-carga-asignatura-cap2}
\end{table}

En las columnas de la tabla \ref{tabla-carga-asignatura-cap2} se muestran: 
el nombre de la asignatura, el año escolar en el cual se imparte, la facultad a la que 
pertenece el año escolar, la cantidad total de horas clase, la cantidad de grupos 
de conferencias y de clases prácticas, y la distribución de horas clase en 
conferencias y clases prácticas. 
Por ejemplo, la primera fila de la tabla indica que la asignatura Optimización Matemática I, 
que se imparte a los estudiantes de tercer año de la carrera 
Matemática, tiene un total de 64 horas clase, que se distribuyen en 32 horas de 
conferencias y 32 horas de clases prácticas,
y que solo existe un grupo para las conferencias y para las clases prácticas.


% Una vez que se definan las cargas de las asignaturas, el jefe del departamento
% comienza el proceso de asignación. Se desea que el sistema muestre
% los profesores que están disponibles para 
% cubrir las actividades de clase de las asignaturas (conferencias, clases prácticas).

A partir de esta información, el jefe del departamento comienza el proceso de asignación.
Se desea que el sistema muestre los profesores que están disponibles para 
cubrir la carga de las asignaturas.
En la tabla \ref{tabla-asignación-cap2} se muestra una posible asignación docente 
para este ejemplo.

\begin{table}[H]
    \centering
    \begin{tabular}{ | c | c | c | c |}
      \hline
      \thead{Asignatura} & \thead{Horas} & \thead{Grupos} & \thead{Profesores}\\
      \hline
      Optimización Matemática I &  64  & 1/1 & \makecell{C: Aymée Marrero (32) \\ CP: Gemayqzel Bouza(32)} \\
      \hline
      Modelos de Optimización I   &  64   &  1/2 & \makecell{C: Aymée Marrero(16) \\ C: Fernando Rodríguez (16) \\ CP: Daniela González(32) \\ CP: Camila Pérez(32)}    \\ 
      \hline
      Estadística                 &  80   &  1/2 &  \makecell{C: Elianys García (48) \\ CP: Ernesto Borrego(32) \\ CP: Elianys García(32)} \\  
      \hline
    \end{tabular}
    \caption{Posible asignación de docencia.}
    \label{tabla-asignación-cap2}
\end{table}

En la primera fila de la tabla \ref{tabla-asignación-cap2}, se indica 
que las 32 horas de conferencia de la asignatura Optimización Matemática I,
se le asignaron a la profesora Aymée Marrero y las 32 horas de clases prácticas a la profesora Gemayqzel Bouza.

Cada vez que el jefe de departamento realice una 
asignación, se desea que se actualice la carga docente que tendrá en el semestre el profesor asignado, de forma que se conozca la carga 
de todos los profesores durante el proceso de asignación.
Para la asignación que se muestra en la tabla \ref{tabla-asignación-cap2}, 
el sistema de gestión debe mostrar las siguientes cargas docentes:

\begin{multicols}{2}
    \begin{itemize}
        \item Aymée Marrero: 48 horas
        \item Gemayqzel Bouza: 32 horas 
        \item Fernando Rodríguez: 16 horas 
        \item Daniela González: 32 horas 
        \item Camila Pérez: 32 horas
        \item Elianys García: 80 horas
        \item Ernesto Borrego: 32 horas 
    \end{itemize}
\end{multicols}

En este caso se puede apreciar que la carga docente de la profesora 
Elianys García es mayor que la del resto de los profesores, lo cual se 
quisiera evitar con la implementación de esta funcionalidad. 


Por otra parte, se quiere también que el sistema muestre las asignaturas o 
grupos que faltan por cubrir para cumplir con la planificación de la docencia. 
Además, una vez concluida la asignación, se desea poder exportar la información de la docencia a un documento con el 
formato que se utiliza actualmente en la facultad. 


En la próxima sección se describen las funcionalidades que se desean incorporar para el 
proceso de planificación de las tesis.


\section{Planificación de las tesis}\label{tesis:cap2}
El proceso de planificación de las tesis se puede descomponer en dos subprocesos principales: 
la confección de los tribunales y la programación de las defensas.
Para la planificación de las tesis, se desea disponer de las siguientes funcionalidades.



\begin{itemize}
    \item Cada vez que se agregue una tesis en el sistema, crear un tribunal sin definir para la misma.
    \item Conocer, durante el proceso de confección de los tribunales de tesis, la cantidad de tribunales 
    en los que participa cada profesor.
    \item Una vez se confeccione el tribunal de una tesis, crear un horario sin definir para la defensa de la misma.
    \item Una vez estén definidos los horarios, se desea conocer en cuáles, un determinado profesor no está libre (por participaciones en defensas). 
    Con el objetivo de facilitar cambios necesarios en la programación de los actos de defensa.  
    \item Exportar la confección de los tribunales de tesis a un documento con un formato preestablecido.
    \item Exportar la programación de las defensas de tesis a un documento con un formato preestablecido.
\end{itemize}

% A continuación se describen como se desean utilizar las funcionalidades anteriores
% durante el proceso de planificación de tesis.

Para ilustrar las funcionalidades que se desean durante el proceso de 
planificación de las tesis se utilizará un ejemplo.
En la tabla \ref{tabla-tesis-cap2}, se muestran dos tesis.

\begin{table}[H]
    \centering
    \begin{tabular}{ | c | c | c | c |}
      \hline
      \thead{ID} & \thead{Tesis} & \thead{Estudiante} & \thead{Tutores} \\
      \hline 
             1 & \makecell{Simulación y optimización \\ de movimientos de malabares} & Gustavo Despaigne & Fernando Rodríguez  \\
      \hline
             2 & \makecell{Propagación de epidemias \\ mediante modelos basados \\ en metapoblaciones} & Abel Antonio Cruz & \makecell{Angela M. León \\ José A. Mesejo} \\
      \hline
    \end{tabular}
    \caption{Ejemplo de dos tesis}
    \label{tabla-tesis-cap2}
\end{table}

Las columnas de la tabla \ref{tabla-tesis-cap2} muestran: un identificador, el nombre de la tesis, el nombre del autor y 
los nombres de los tutores

El primer paso es la confección de los tribunales para cada tesis. Se debe definir qué
profesor formará parte de cada tribunal y con qué rol. En este ejemplo 
se asume que los tribunales solo se conformarán por un oponente y un presidente.
A continuación se muestran posibles tribunales para las tesis correspondientes a la 
tabla \ref{tabla-tesis-cap2}.


% El encargado de realizar la planificación de las tesis debe confeccionar un tribunal para 
% cada una de ellas. Se desea que el sistema de gestión muestre los profesores de la facultad para 
% la confección de los tribunales. En el curso 2022 los tribunales estarán compuestos por los tutores, 
% un oponente y un presidente, por tanto solo se deben determinar los profesores que asumirán los roles 
% de oponente y presidente. En la figura tal se muestran posibles tribunales para las tesis correspondientes 
% a la tabla \ref{tabla-tesis-cap2}.

\begin{table}[H]
    \centering
    \begin{tabular}{ | c | c | c | c |}
      \hline
      \thead{ID Tesis} & \thead{Tutores} & \thead{Oponente} & \thead{Presidente} \\
      \hline 
             1 & Fernando Rodríguez & Gemayqzel Bouza & Aymée Marrero  \\
      \hline
             2 & \makecell{Angela M. León \\ José A. Mesejo} & Damian Valdés & Gemayqzel Bouza  \\
      \hline
    \end{tabular}
    \caption{Posibles tribunales de tesis}
    \label{tabla-tribunal-tesis-cap2}
\end{table}

Con el objetivo de realizar una distribución equitativa de los profesores en los tribunales de tesis, 
se desea que durante el proceso de confección de los tribunales, se pueda conocer la cantidad en los que participa cada profesor.
En la tabla \ref{tabla-carga-profesores-tribunales}
se muestran las participaciones de los profesores en tribunales 
de acuerdo a la tabla \ref{tabla-tribunal-tesis-cap2}

\begin{table}[H]
    \centering
    \begin{tabular}{ | c | c | c |}
      \hline
      \thead{Profesor} & \thead{Oponente} & \thead{Presidente} \\
      \hline 
             Gemayqzel Bouza & 1 & 1  \\
      \hline
             Aymée Marrero & 0 & 1 \\
      \hline
            Damian Valdés & 1 & 0 \\
      \hline
    \end{tabular}
    \caption{Cantidad de tribunales de tesis en los que participa cada profesor}
    \label{tabla-carga-profesores-tribunales}
\end{table}


Una vez que se definan los tribunales de tesis, el próximo paso es 
relizar la programación de las defensas. 
Se debe definir la fecha, hora y local en donde se realizará
el acto de defensa de cada tesis. En la tabla \ref{tabla-defensa-tesis-cap2} se muestran posibles 
horarios para las defensas de las 
tesis que se muestran en la tabla \ref{tabla-tesis-cap2}.

\begin{table}[H]
    \centering
    \begin{tabular}{ | c | c | c | c |}
      \hline
      \thead{ID Tesis} & \thead{Fecha} & \thead{Hora} & \thead{Local} \\
      \hline 
             1 & 10/12/2022 & 2:00 PM & Aula Posgrado  \\
      \hline
             2 & 10/12/2022 & 3:30 PM & Salón del Decanato \\
      \hline
    \end{tabular}
    \caption{Posible horarios para las defensas de tesis}
    \label{tabla-defensa-tesis-cap2}
\end{table}

Una vez concluido el proceso de planificación de las tesis, 
se desea poder exportar los datos de los tribunales y la programación de las 
defensas a documentos con un formato 
preestablecido. 

Con el objetivo de evitar que una tesis se quede sin planificar, se desea que cada vez que una tesis se agregue al sistema, se genere automáticamente 
un tribunal sin definir y una programación para el acto de defensa, sin definir la fecha, la hora ni el local.

Además se desea conocer para cada profesor los horarios que tiene ocupado a partir de la programación de las defensas, 
por si es necesario realizar algún cambio en la planificación.


En este capítulo se describieron las funcionalidades que se desean incorporar en el sistema 
como parte de la informatización de los procesos de asignación de docencia y planificación de las tesis. 
En el próximo capítulo se describe como se modeló y diseñó la base de datos para representar 
la información que interviene en estos procesos. 


