\chapter{Descripción de las funcionalidades que se desean}\label{chapter:features}

El objetivo principal de este trabajo es la creación de
una herramienta que permita la informatización de los procesos de
asignación de docencia, confección de los tribunales de tesis y planificación de las defensas
de las tesis, que se llevan a cabo en la Facultad de Matemática y Computación
de la Universidad de La Habana.
Se desea además que la herramienta permita la 
posterior integración de los procesos que se abordan en la 
sección \ref{section:funcionamiento de la facultad}.


En las siguientes secciones se describen las funcionalidades
que se desean para cada proceso.

% El objetivo principal de este trabajo es la creación de
% una herramienta que permita la informatización de los procesos
% de administración y planificación que se llevan a cabo en la Facultad de Matemática y Computación
% de La Universidad de La Habana. 
% Se desea implementar un sistema de 
% gestión para realizar los procesos de asignación de docencia y 
% confección de los tribunales de tesis, y que permita 
% la posterior integración de los procesos que se abordan en la 
% sección \ref{section:funcionamiento de la facultad}.












% La primera funcionalidad que se desea es la informatización
% de los datos que intervienen en estos procesos.
% Una vez se tengan los datos se desea realizar la 
% asignación de docencia de un departamento, así como la
% confección de los tribunales de tesis. A continuación 
% se describen estos procesos.


\section{Asignación de docencia}
Se desea implementar una mecanismo que permita realizar la asignación de docencia 
facilitando el acceso a toda la información que interviene en este proceso. A continuación 
se describen las principales funcionalidades que se desean:


\begin{itemize}
    \item Realizar la asignación de docencia en un departamento, mostrando solo los datos relevantes 
     para el mismo. Por ejemplo, cuando se realice la asignación de docencia para el departamento de Matemática Aplicada, solo 
     se mostrarán las asignaturas y profesores del mismo.
    \item Conocer la carga docente de los profesores durante el proceso de asignación.
    \item Conocer las asignaturas que se deben impartir en un semestre a partir de un curso escolar y un período de tiempo. 
          Por ejemplo, se desea saber las asignaturas a impartir en el curso 2021-2022 en el período de tiempo enero-julio.
    \item Realizar filtrados o búsquedas a partir de los datos que intervienen en la asignación 
          de docencia. Por ejemplo, saber las asignaturas que debe impartir un profesor en un semestre dado.
    \item Exportar la asignación de docencia a un documento.

\end{itemize}


\section{Confección de los tribunales de tesis y planificación de las defensas de tesis}
Se desea implementar un mecanismo que permita la confección de los tribunales de tesis y 
la planificación de las defensas de las tesis. Además se desean algunas funcionalidades 
relacionadas con estos procesos, a continuación se describen.

\begin{itemize}
    \item Realizar la confección de los tribunales de tesis a partir de las tesis que se deban realizar como ejercicio de 
    culminación en un curso escolar.
    \item Conocer la cantidad de tribunales de los que forman parte los profesores durante el proceso 
    de confección de los tribunales de tesis.
    \item Realizar filtrados o búsquedas a partir de los datos que intervienen en la confección 
          de tribunales de tesis. Por ejemplo, saber en cuáles tribunales participa un profesor y con qué rol (secretario, oponente, presidente).
    \item Realizar la planificación de las defensas de tesis a partir de la confección de los tribunales de tesis.
    \item Realizar filtrados o búsquedas a partir de los datos que intervienen en la planificación 
    de las defensas de tesis. Por ejemplo, saber dado una fecha cuáles tesis se defienden en la misma o el local 
    en donde se llevará a cabo el ejercicio.
    \item Exportar la confección de los tribunales de tesis a un documento.
    \item Exportar la planificación de las defensas de tesis a un documento.

\end{itemize}


