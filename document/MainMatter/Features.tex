\chapter{Descripción de las funcionalidades que se desean}\label{chapter:features}
El objetivo principal de este trabajo es la creación de
una herramienta que permita la informatización de los procesos
de administración y planificación que se llevan a cabo en la Facultad de Matemática y Computación
de La Universidad de La Habana. 
Se desea implementar un sistema de 
gestión para realizar los procesos de asignación de docencia y 
confección de los tribunales de tesis, y que permita 
la posterior integración de los procesos que se abordan en la 
sección \ref{section:funcionamiento de la facultad}.





Una de las funcionalidades principales que se desea es la informatización
de los datos relativos a la facultad. La digitalización de datos como 
los planes de estudios, departamentos de la facultad, profesores, asignaturas 
y cualquier otra información que intervenga en los procesos que se modelen en el sistema 
de gestión puede ser útil tanto para el desarrollo de herramientas informativas como
para el análisis y procesamiento de los datos con el objetivo de computar métricas y 
estádisticas (esto esta malito).






% La primera funcionalidad que se desea es la informatización
% de los datos que intervienen en estos procesos.
% Una vez se tengan los datos se desea realizar la 
% asignación de docencia de un departamento, así como la
% confección de los tribunales de tesis. A continuación 
% se describen estos procesos.


\section{Asignación de docencia}
Se desea implementar un mecanismo que permita realizar
la asignación de docencia .
Se quiere además la generación de un documento
CSV que agrupe esta información.


\section{Confección de los tribunales de tesis}
Se desea implementar la funcionalidad de creación
de tribunales de tesis, dada una tesis poder asignar
los profesores que conformarán el tribunal en los
roles de secretario, presidente y oponente, así como
un lugar y fecha para la defensa de la misma. De igual
forma se desea la generación de un documento CSV con
esta información.
