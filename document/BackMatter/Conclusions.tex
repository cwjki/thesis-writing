\begin{conclusions}
    En la facultad de Matemática y Computación de la Universidad de La Habana (MATCOM)
se realizan un conjunto de procesos para administrar y planificar las actividades docentes. 
La mayoría de estas tareas se llevan a cabo de forma manual por los profesores de la facultad, lo cual representa
un derroche de tiempo que se pudiera dedicar a la investigación.
Dos de estos procesos son la asignación de docencia y la planificación de las tesis. 

En este trabajo se diseña e implementa un sistema de gestión para informatizar 
la asignación de docencia y la planificación de las tesis, creando una plataforma que permite 
la posterior integración de otros procesos.

    
Se desarrolló una aplicación web teniendo en cuenta el flujo de trabajo que se sigue actualmente para 
llevar a cabo ambos procesos, con el objetivo de facilitar la utilización del sistema como herramienta
para realizarlos.

La información necesaria para realizar la asignación de docencia y 
las planificaciones de las tesis  
se almacena en una base de datos.
Para el diseño de la base de datos se contempló la necesidad de integrar 
posteriormente la informatización y/o automatización del resto de los procesos 
administrativos de la facultad.

Además, se implementó la funcionalidad de exportar la
información correspondiente a la asignación docente,
la confección de los tribunales y la programación de las defensas de las tesis, 
con el mismo formato utilizado actualmente para la planificación de estos procesos.


Para comprobar el funcionamiento del sistema,
se reprodujo la asignación de docencia para el departamento 
de Matemática Aplicada correspondiente al período 
enero--julio del curso 2022, y se simuló la confección de los tribunales 
y la programación de las defensas de las tesis correspondientes
a este curso. 

El sistema que se propone representa una base para la informatización de la gestión de la facultad,
ya que, aunque de momento solo se informatizan la asignación de docencia
y la planificación de las tesis, se diseñó
con la perspectiva de integrar en un único sistema todas las tareas
administrativas de la facultad. \\

Algunas de las recomendaciones que se proponen son las
siguientes:
\begin{itemize}
    \item Implementar un mecanismo para automatizar la asignación de docencia.
    \item Implementar un mecanismo para automatizar la planificación de las tesis contemplando 
    las investigaciones que se han realizado en la facultad relacionadas con este tema, como 
    \cite{tribunales}.
    \item Incorporar al sistema otros procesos de gestión de la facultad.
    \item Agregar la información de la docencia externa, correspondiente a las facultades a las que se les brinda servicio.
    \item Agregar las áreas de investigación de los profesores para mejorar el proceso de confección de los tribunales.
\end{itemize}

\end{conclusions}



